 % Всякое, чтобы работало - все библиотеки
 \documentclass[a4paper, 12pt]{article}
 
 \usepackage[T2A]{fontenc}
 \usepackage[russian, english]{babel}
 \usepackage[utf8]{inputenc}
 \usepackage{subfiles}
 \usepackage{ucs}
 \usepackage{textcomp}
 \usepackage{array}
 \usepackage{indentfirst}
 \usepackage{amsmath}
 \usepackage{amssymb}
 \usepackage{enumerate}
 \usepackage[margin=1.5cm]{geometry}
 \usepackage{authblk}
 \usepackage{tikz}
 \usepackage{icomma}
 \usepackage{gensymb}
 \usepackage{nicematrix, tikz}
  
 % Всякие мат штуки дополнительные
  
 \newcommand{\F}{\mathbb{F}}
 \newcommand{\di}{\frac}
 \renewcommand{\C}{\mathbb{C}}
 \newcommand{\N} {\mathbb{N}}
 \newcommand{\Z} {\mathbb{Z}}
 \newcommand{\R} {\mathbb{R}}
 \newcommand{\Q}{\mathbb{Q}}
 \newcommand{\ord} {\mathop{\rm ord}}
 \newcommand{\Ima}{\mathop{\rm Im}}
 \newcommand{\Rea}{\mathop{\rm Re}}
 \newcommand{\rk}{\mathop{\rm rk}}
 \newcommand{\arccosh}{\mathop{\rm arccosh}}
 \newcommand{\lker}{\mathop{\rm lker}}
 \newcommand{\rker}{\mathop{\rm rker}}
 \newcommand{\tr}{\mathop{\rm tr}}
 \newcommand{\St}{\mathop{\rm St}}
 \newcommand{\Mat}{\mathop{\rm Mat}}
 \newcommand{\grad}{\mathop{\rm grad}}
 \DeclareMathOperator{\spec}{spec}
 \renewcommand{\baselinestretch}{1.5}
 \everymath{\displaystyle}
  
 % Всякое для ускорения
 \renewcommand{\r}{\right}
 \renewcommand{\l}{\left}
 \newcommand{\Lra}{\Leftrightarrow}
 \newcommand{\ra}{\rightarrow}
 \newcommand{\la}{\leftarrow}
 \newcommand{\sm}{\setminus}
 \newcommand{\lm}{\lambda}
 \newcommand{\Sum}[2]{\overset{#2}{\underset{#1}{\sum}}}
 \newcommand{\Lim}[2]{\lim\limits_{#1 \rightarrow #2}}
 \newcommand{\p}[2]{\frac{\partial #1}{\partial #2}}
  
 % Заголовки
 \newcommand{\task}[1] {\noindent \textbf{Задача #1.} \hfill}
 \newcommand{\note}[1] {\noindent \textbf{Примечание #1.} \hfill}
  
 % Пространтсва для задач
 \newenvironment{proof}[1][Доказательство]{%
 \begin{trivlist}
     \item[\hskip \labelsep {\bfseries #1:}]
     \item \hspace{15pt}
     }{
     $ \hfill\blacksquare $
 \end{trivlist}
 \hfill\break
 }
 \newenvironment{solution}[1][Решение]{%
 \begin{trivlist}
     \item[\hskip \labelsep {\bfseries #1:}]
     \item \hspace{15pt}
     }{
 \end{trivlist}
 }
  
 \newenvironment{answer}[1][Ответ]{%
 \begin{trivlist}
     \item[\hskip \labelsep {\bfseries #1:}] \hskip \labelsep
     }{
 \end{trivlist}
 \hfill
 }
 \title{Дз по линейной алгебре 2 Смирнов Тимофей 236 ПМИ}
 \author{Тимофей Смирнов}
 \date{September 2023}
 
 \begin{document}
    {\center \bf \large ДЗ по линейной алгебре 5 Смирнов Тимофей 236 ПМИ}
    \\ \textbf{№ 1} 
    \\
    \\ 1.1
        $ \begin{vmatrix}
            5 & 4 \\
            7 & 3 \\
        \end{vmatrix} = (1) \cdot 5 \cdot 3 + (-1) \cdot 4 \cdot 7 = -13$
    \\
    \\ 1.2 $ \begin{vmatrix}
        2 & 1 & 3 \\
        5 & 3 & 2 \\
        1 & 4 & 3 \\
    \end{vmatrix} = (1) \cdot 2 \cdot 3 \cdot 3 + (1) \cdot 5 \cdot 4 \cdot 3 + (1) \cdot 1 \cdot 2 \cdot 1 + (-1) \cdot 3 \cdot 3 \cdot 1 + (-1) \cdot 2 \cdot 2 \cdot 4 + (-1) \cdot 5 \cdot 3 \cdot 1 = 40$
    \\
    \\ 1.3 $\begin{vmatrix}
        0 & 2 & 2 \\
        2 & 0 & 2 \\
        2 & 2 & 0 \\
    \end{vmatrix} = (1) \cdot 0 \cdot 0 \cdot 0 + (1) \cdot 2 \cdot 2 \cdot 2 + (1) \cdot 2 \cdot 2 \cdot 2 + (-1) \cdot 0 \cdot 2 \cdot 2 + (-1) \cdot 2 \cdot 2 \cdot 0 + (-1) \cdot 0 \cdot 2 \cdot 2 = 16$
    \\
    \\ 1.4 $\begin{vmatrix}
        n + 1 & n \\
        n & n - 1 \\
    \end{vmatrix} = (1) \cdot (n + 1) \cdot (n - 1) + (-1) \cdot n \cdot n = -1$
    \\
    \\ \textbf{№ 2}
    \\
    \\ Нам нужно найти такие i и k, чтобы перестановка $\begin{pmatrix}1 & 2 & 3 & 4 & 5 & 6 \\ ? & 1 & 3 & 6 & ? & 2 \end{pmatrix}$ имела знак -1. При этом i может быть равно 1 или 5, как и k, но равняться одному числу они не могут. Иначе перестановка будет задана некоррекнтно.
    \\
    \\ Ннужно перебрать только 2 варианта:
    \\
    \par \ \ \ \ 1). $\begin{pmatrix}1 & 2 & 3 & 4 & 5 & 6 \\ 5 & 1 & 3 & 6 & 4 & 2 \end{pmatrix}$ Здесь $i = 1, k = 5$. В этой перестановке 8 инверсий, она четная, следовательно, такие i и k нам не подходят.
    \\
    \par \ \ \ \ 2). $\begin{pmatrix}1 & 2 & 3 & 4 & 5 & 6 \\ 4 & 1 & 3 & 6 & 5 & 2 \end{pmatrix}$ Здесь $i = 5, k = 1$. В этой перестановке 7 инверсий, следовательно $i = 5, k = 1$ нам подходят.
    \\
    \\ \textbf{Ответ: } $i = 5, \ \  k = 1$.
    \\
    \\ \textbf{№ 3} \textbf{Как изменится определитель матрицы, если переставить ее столбцы в обратном порядке?}
    \\
    \\ По свойству определителя, если поменять две строки или столбца матрицы местами, то определитель поменяет знак, рассмотрим, сколько таких перестановок нужно, чтобы переставить столбцы матрицы в обратном порядке.
    \\
    \\ 1). Пусть в матрице n столбцов и n четно, тогда достаточно поменять местами 1 и n-ю строку, потому 2ю и (n - 1)ю и так далее. Всего получится $\frac{n}{2}$ перестановок, то есть знак поменяется $\frac{n}{2}$ раз, если это число четное, то знак не изменится, если не четное - изменится.
    \\
    \\ 2). Пусть теперь n нечетное, тогда в наших перестановках поменяется местами (n - 1) строка, то есть все, кроме средней строки, получаем $\frac{(n - 1)}{2}$ замену. Если это число будет четным, то знак определителя не поменяется, иначе изменится на противоположный.
    \\
    \\
    \\ \textbf{№ 4 Как изменится определитель матрицы, если повернуть ее на 90 градусов по часовой стрелке?}
    \\
    \\ Чтобы размернуть матрицу на 90 градусов по часовой стрелке достаточно транспонировать ее, а потом применить к ней действия из предыдущей задачи, то есть переставить ее столбцы в обратном порядке.
    \\
    \\ При транспонировании определитель матрицы не меняется, следовательно он зависит только от количества столбцов в матрице.
    \\
    \\ 1). Пусть в матрице четное число столбцов n, тогда после транспонирвания нам достаточно будет совершить $\frac{n}{2}$ операций, если число $\frac{n}{2}$ будет четным, то знак определителя не поменяется, иначе - изменится на противоположный.
    \\
    \\ 2). Пусть в матрице нечетное число столбцов n, тогда после транспонирования нам достаточно будет совершить $\frac{n - 1}{2}$ операцию, если это число $\frac{n - 1}{2}$ будет четным, то определитель не изменится, иначе - изменит знак.
    \\
    \\
    \\
    \\
    \\
    \\ \textbf{№ 5} \ \ \ \textbf{Как изменится определитель матрицы, если из каждой строки, кроме последней, вычесть последнюю строку, а из последней строкив вычесть исходную первую стоку?}
    \\
    \\ Пусть в нашей матрице k строк, после вычитания последующей строки из строк с 1й по (k - 1)-ю определитель не изменится (так как при прибавлении к одной строке или столбцу матрицы другого столбца или строки, умноженных на скаляр, определитель не меняется).
    \\
    \\ Обозначим первую строку как $A_1$, вторую $A_2$ и тд, последняя строка будет $A_k$, тогда последнюю строку новой матрицы можно будет представить в виде $A_k - A_1$.
    \\
    \\ По свойству определителя $\begin{vmatrix}A_1 - A_2 \\ A_2 - A_3 \\ \dots \\ A_k + (-A_1)\end{vmatrix} = \begin{vmatrix}A_1 - A_2 \\ A_2 - A_3 \\ \dots \\ A_k\end{vmatrix} + \begin{vmatrix}A_1 - A_2 \\ A_2 - A_3 \\ \dots \\ -A_1\end{vmatrix}$. 
    \\ По свойству определителя $\begin{vmatrix}A_1 - A_2 \\ A_2 - A_3 \\ \dots \\ -A_1\end{vmatrix} = -\begin{vmatrix}A_1 - A_2 \\ A_2 - A_3 \\ \dots \\ A_1\end{vmatrix} \ \Rightarrow \ \begin{vmatrix}A_1 - A_2 \\ A_2 - A_3 \\ \dots \\ A_k + (-A_1)\end{vmatrix} = \begin{vmatrix}A_1 - A_2 \\ A_2 - A_3 \\ \dots \\ A_k\end{vmatrix} - \begin{vmatrix}A_1 - A_2 \\ A_2 - A_3 \\ \dots \\ A_1\end{vmatrix}$
    \\
    \\ По свойству определителя $\begin{vmatrix}A_1 - A_2 \\ A_2 - A_3 \\ \dots \\ A_1\end{vmatrix} = \begin{vmatrix}A_1 - A_2 \\ A_2 - A_3 \\ \dots \\ A_k + (A_1 - A_2) + (A_2 - A_3) + ... + (A_{k - 1} - A_k)\end{vmatrix} = \begin{vmatrix}A_1 - A_2 \\ A_2 - A_3 \\ \dots \\ A_k\end{vmatrix}$ (мы прибавили все строки кроме последней к последней строке).
    \\
    \\ Получаем $\begin{vmatrix}A_1 - A_2 \\ A_2 - A_3 \\ \dots \\ A_k + (-A_1)\end{vmatrix} = \begin{vmatrix}A_1 - A_2 \\ A_2 - A_3 \\ \dots \\ A_k\end{vmatrix} - \begin{vmatrix}A_1 - A_2 \\ A_2 - A_3 \\ \dots \\ A_1\end{vmatrix} = \begin{vmatrix}A_1 - A_2 \\ A_2 - A_3 \\ \dots \\ A_k\end{vmatrix} - \begin{vmatrix}A_1 - A_2 \\ A_2 - A_3 \\ \dots \\ A_k\end{vmatrix} = 0$
    \\
    \\ \textbf{Ответ: } определитель такой матрицы будет равен 0.
    \\
    \\
    \\ \textbf{№ 6} Посчитайте определитель матрицы, приведя ее к ступенчатому виду
    \\
    \\ \textbf{6.1}  \ \ \ \ \ Пусть $A = \begin{vmatrix}0 & 0 & 4 \\ -2 & -11 & 7 \\ 0 & 80 & -20\end{vmatrix}$.
    \\
    \\ Рассмотрим матрицу $\begin{pmatrix}0 & 0 & 4 \\ -2 & -11 & 7 \\ 0 & 80 & -20\end{pmatrix}$ и проведем с ней элементарные преобразования, приведя ее к улучшенному ступенчатому виду:
    \\
    \\ 1. $\begin{pmatrix}0 & 0 & 4 \\ -2 & -11 & 7 \\ 0 & 80 & -20\end{pmatrix} \to \begin{pmatrix}0 & 80 & -20 \\-2 & -11 & 7 \\ 0 & 0 & 4\end{pmatrix} \ \Rightarrow \ A \to -A$ (параллельно мы показываем, что происходит с определителем данной матрицы)
    \\
    \\ 2. $\begin{pmatrix}0 & 80 & -20 \\-2 & -11 & 7 \\ 0 & 0 & 4\end{pmatrix} \to \begin{pmatrix}-2 & -11 & 7 \\ 0 & 80 & -20 \\ 0 & 0 & 4\end{pmatrix} \ \Rightarrow \ -A \to A$
    \\\\
    \\ 3. $\begin{pNiceArray}{ccc}[last-col,margin]-2 & -11 & 7 & \times (-\frac{1}{2})\\ 0 & 80 & -20 & \\ 0 & 0 & 4 & \end{pNiceArray} \to \begin{pNiceArray}{ccc}[last-col,margin]1 & \frac{11}{2} & -\frac{7}{2} & \\ 0 & 80 & -20 & \\ 0 & 0 & 4 & \end{pNiceArray} \ \Rightarrow \ A \to -\frac{1}{2}A$
    \\
    \\
    \\ 4. $\begin{pNiceArray}{ccc}[last-col,margin]1 & \frac{11}{2} & -\frac{7}{2} & \\ 0 & 80 & -20 & \times \frac{1}{80} \\ 0 & 0 & 4 & \end{pNiceArray} \to \begin{pNiceArray}{ccc}[last-col,margin]1 & \frac{11}{2} & -\frac{7}{2} & \\ 0 & 1 & -\frac{1}{4} & \\ 0 & 0 & 4 & \end{pNiceArray} \ \Rightarrow \ -\frac{1}{2}A \to -\frac{1}{160}A$
    \\
    \\
    \\ 5. $\begin{pNiceArray}{ccc}[last-col,margin]1 & \frac{11}{2} & -\frac{7}{2} & \\ 0 & 1 & -\frac{1}{4} & \\ 0 & 0 & 4 & \times \frac{1}{4}\end{pNiceArray} \to \begin{pNiceArray}{ccc}[last-col,margin]1 & \frac{11}{2} & -\frac{7}{2} & \\ 0 & 1 & -\frac{1}{4} & \\ 0 & 0 & 1 & \end{pNiceArray} \ \Rightarrow \ -\frac{1}{160}A = -\frac{1}{640}A$
    \\
    \\ Мы привели матрицу к верхнетреугольному виду. Мы получаем уравнение $-\frac{1}{640}A = 1 \ \Rightarrow \ A = -640$
    \\
    \\ \textbf{Ответ: } -640
    \\
    \\
    \\ \textbf{6.2} Пусть $A = \begin{vmatrix}1 & 2 & 3 \\ 4 & 5 & 6 \\ 7 & 8 & 9 \end{vmatrix}$
    \\
    \\ Рассмотрим матрицу $\begin{pNiceArray}{ccc}[last-col,margin]1 & 2 & 3 & \\ 4 & 5 & 6 & \\ 7 & 8 & 9 & \end{pNiceArray}$ и проведем с ней преобразования, приведя ее к ступенчатому виду для нахождения определителя.
    \\
    \\ 1. $\begin{pNiceArray}{ccc}[last-col,margin]1 & 2 & 3 & \\ 4 & 5 & 6 & -4(1) \\ 7 & 8 & 9 & -7(1)\end{pNiceArray} \to \begin{pNiceArray}{ccc}[last-col,margin]1 & 2 & 3 & \\ 0 & -3 & -6 & \\ 0 & -6 & -12 & \end{pNiceArray} \ \Rightarrow \ A \to A$ (по свойству определителя)
    \\
    \\ 2. $\begin{pNiceArray}{ccc}[last-col,margin]1 & 2 & 3 & \\ 0 & -3 & -6 & \times (-\frac{1}{3})\\ 0 & -6 & -12 & \end{pNiceArray} \to \begin{pNiceArray}{ccc}[last-col,margin]1 & 2 & 3 & \\ 0 & 1 & 2 & \\ 0 & -6 & -12 & \end{pNiceArray} \ \Rightarrow \ A \to -\frac{1}{3}A$
    \\
    \\ 3. $\begin{pNiceArray}{ccc}[last-col,margin]1 & 2 & 3 & \\ 0 & 1 & 2 & \\ 0 & -6 & -12 & +6(1)\end{pNiceArray} \to \begin{pNiceArray}{ccc}[last-col,margin]1 & 2 & 3 & \\ 0 & 1 & 2 & \\ 0 & 0 & 0 & \end{pNiceArray} \ \Rightarrow \ A \to A$, но так как мы плучили матрицу с углом нулей, то $A = 1 \cdot 1 \cdot 0 = 0 \ \Rightarrow \ $ определитель матрицы $\begin{pNiceArray}{ccc}[last-col,margin]1 & 2 & 3 & \\ 4 & 5 & 6 & \\ 7 & 8 & 9 & \end{pNiceArray}$ равен 0.
    \\
    \\
    \\ \textbf{Ответ: } 0
    \\
    \\
    \\ \textbf{6.3} \ \ \ \ \ Пусть $A = \begin{vmatrix}1001 & 1002 & 1003 & 1004 \\ 1002 & 1003 & 1001 & 1002 \\ 1001 & 1001 & 1001 & 999 \\ 1001 & 1000 & 998 & 999 \end{vmatrix}$
    \\\\
    \\ Рассмотрим матрицу $\begin{pNiceArray}{cccc}[last-col,margin]1001 & 1002 & 1003 & 1004 & \\ 1002 & 1003 & 1001 & 1002 & \\ 1001 & 1001 & 1001 & 999 & \\ 1001 & 1000 & 998 & 999 & \\ \end{pNiceArray}$
    \\
    \\ 1. $\begin{pNiceArray}{cccc}[last-col,margin]1001 & 1002 & 1003 & 1004 & \\ 1002 & 1003 & 1001 & 1002 & -1(1)\\ 1001 & 1001 & 1001 & 999 & \\ 1001 & 1000 & 998 & 999 & \\ \end{pNiceArray} \to \begin{pNiceArray}{cccc}[last-col,margin]1001 & 1002 & 1003 & 1004 & \\ 1 & 1 & -2 & -2 & \\ 1001 & 1001 & 1001 & 999 & \\ 1001 & 1000 & 998 & 999 & \\ \end{pNiceArray} \ \Rightarrow \ A \to A$
    \\
    \\
    \\ 2. $\begin{pNiceArray}{cccc}[last-col,margin]1001 & 1002 & 1003 & 1004 & -1(3)\\ 1 & 1 & -2 & -2 & \\ 1001 & 1001 & 1001 & 999 & \\ 1001 & 1000 & 998 & 999 & \\ \end{pNiceArray} \to \begin{pNiceArray}{cccc}[last-col,margin]0 & 1 & 2 & 5 & \\ 1 & 1 & -2 & -2 & \\ 1001 & 1001 & 1001 & 999 & \\ 1001 & 1000 & 998 & 999 & \\ \end{pNiceArray} \ \Rightarrow \ A \to A$
    \\
    \\
    \\ 3. $\begin{pNiceArray}{cccc}[last-col,margin]0 & 1 & 2 & 5 & \\ 1 & 1 & -2 & -2 & \\ 1001 & 1001 & 1001 & 999 & -1(4)\\ 1001 & 1000 & 998 & 999 & \\ \end{pNiceArray} \to \begin{pNiceArray}{cccc}[last-col,margin]0 & 1 & 2 & 5 & \\ 1 & 1 & -2 & -2 & \\ 0 & 1 & 3 & 0 & \\ 1001 & 1000 & 998 & 999 & \\ \end{pNiceArray} \ \Rightarrow \ A \to A$
    \\
    \\
    \\ 4. $\begin{pNiceArray}{cccc}[last-col,margin]0 & 1 & 2 & 5 & \\ 1 & 1 & -2 & -2 & \\ 0 & 1 & 3 & 0 & \\ 1001 & 1000 & 998 & 999 & -1001(2)\\ \end{pNiceArray} \to \begin{pNiceArray}{cccc}[last-col,margin]0 & 1 & 2 & 5 & \\ 1 & 1 & -2 & -2 & \\ 0 & 1 & 3 & 0 & \\ 0 & -1 & 3000 & 3001 & \\ \end{pNiceArray} \ \Rightarrow \ A \to A$
    \\
    \\
    \\ 5. $\begin{pNiceArray}{cccc}[last-col,margin]0 & 1 & 2 & 5 & \\ 1 & 1 & -2 & -2 & \\ 0 & 1 & 3 & 0 & \\ 0 & -1 & 3000 & 3001 & \\ \end{pNiceArray} \to \begin{pNiceArray}{cccc}[last-col,margin]1 & 1 & -2 & -2 & \\ 0 & 1 & 2 & 5 & \\ 0 & 1 & 3 & 0 & \\ 0 & -1 & 3000 & 3001 & \\ \end{pNiceArray} \ \Rightarrow \ A \to -A$
    \\
    \\
    \\ 6-7. $\begin{pNiceArray}{cccc}[last-col,margin]1 & 1 & -2 & -2 & \\ 0 & 1 & 2 & 5 & +1(4)\\ 0 & 1 & 3 & 0 & +1(4)\\ 0 & -1 & 3000 & 3001 & \\ \end{pNiceArray} \to \begin{pNiceArray}{cccc}[last-col,margin]1 & 1 & -2 & -2 & \\ 0 & 0 & 3002 & 3006 & \\ 0 & 0 & 3003 & 3001 & \\ 0 & -1 & 3000 & 3001 & \\ \end{pNiceArray} \ \Rightarrow \ -A \to -A$
    \\
    \\
    \\ 8. $\begin{pNiceArray}{cccc}[last-col,margin]1 & 1 & -2 & -2 & \\ 0 & 0 & 3002 & 3006 & \\ 0 & 0 & 3003 & 3001 & \\ 0 & -1 & 3000 & 3001 & \times (-1)\\ \end{pNiceArray} \to \begin{pNiceArray}{cccc}[last-col,margin]1 & 1 & -2 & -2 & \\ 0 & 0 & 3002 & 3006 & \\ 0 & 0 & 3003 & 3001 & \\ 0 & 1 & -3000 & -3001 & \\ \end{pNiceArray} \ \Rightarrow \ -A \to A$
    \\
    \\
    \\ 9. $\begin{pNiceArray}{cccc}[last-col,margin]1 & 1 & -2 & -2 & \\ 0 & 0 & 3002 & 3006 & \\ 0 & 0 & 3003 & 3001 & \\ 0 & 1 & -3000 & -3001 & \\ \end{pNiceArray} \to \begin{pNiceArray}{cccc}[last-col,margin]1 & 1 & -2 & -2 & \\ 0 & 1 & -3000 & -3001 & \\ 0 & 0 & 3003 & 3001 & \\ 0 & 0 & 3002 & 3006 & \\ \end{pNiceArray} \ \Rightarrow \ A \to -A$
    \\
    \\
    \\ 10. $\begin{pNiceArray}{cccc}[last-col,margin]1 & 1 & -2 & -2 & \\ 0 & 1 & -3000 & -3001 & \\ 0 & 0 & 3003 & 3001 & -1(4)\\ 0 & 0 & 3002 & 3006 & \\ \end{pNiceArray} \to \begin{pNiceArray}{cccc}[last-col,margin]1 & 1 & -2 & -2 & \\ 0 & 1 & -3000 & -3001 & \\ 0 & 0 & 1 & -5 & \\ 0 & 0 & 3002 & 3006 & \\ \end{pNiceArray} \ \Rightarrow \ -A \to -A$
    \\
    \\
    \\ 11. $\begin{pNiceArray}{cccc}[last-col,margin]1 & 1 & -2 & -2 & \\ 0 & 1 & -3000 & -3001 & \\ 0 & 0 & 1 & -5 & \\ 0 & 0 & 3002 & 3006 & -3002(3)\\ \end{pNiceArray} \to \begin{pNiceArray}{cccc}[last-col,margin]1 & 1 & -2 & -2 & \\ 0 & 1 & -3000 & -3001 & \\ 0 & 0 & 1 & -5 & \\ 0 & 0 & 0 & 18016 & \\ \end{pNiceArray} \ \Rightarrow \ -A \to -A$
    \\
    \\
    \\ Получаем уравнение $-A = 1 \cdot 1 \cdot 1 \cdot 18016 \ \Rightarrow \ A = -18016$
    \\
    \\ \textbf{Ответ: } -18016
    \\
    \\
    \\ \textbf{№ 7} Вычислить определитель $\ \ \ \begin{vmatrix}1 & 2 & 175 & -38 \\ 3 & 4 & 137 & -91 \\ 0 & 0 & 5 & -1 \\ 0 & 0 & 3 & -2 \end{vmatrix}$
    \\
    \\
    \\ Заметим, что данная нам матрица имеет угол нулей, пусть $A = \begin{vmatrix}1 & 2 & 175 & -38 \\ 3 & 4 & 137 & -91 \\ 0 & 0 & 5 & -1 \\ 0 & 0 & 3 & -2 \end{vmatrix}, $ 
    \\ тогда $A = \det \begin{vmatrix}1 & 2 \\ 3 & 4\end{vmatrix} \cdot \det \begin{vmatrix}5 & -1 \\ 3 & -2\end{vmatrix} = (1 \cdot 4 - 2 \cdot 3) \cdot (5 \cdot (-2) - 3 \cdot (-1)) = 14$
    \\
    \\ \textbf{Ответ: } 14
    \\
    \\
    \\ \textbf{№ 8} Раскладывая по строкам и/или столбцам, вычислить $\begin{vmatrix} x & y & 0 & 0 \\ 0 & x & y & 0 \\ 0 & 0 & x & y \\ y & 0 & 0 & x \end{vmatrix}$
    \\
    \\
    \\ Пусть $A = \begin{vmatrix} x & y & 0 & 0 \\ 0 & x & y & 0 \\ 0 & 0 & x & y \\ y & 0 & 0 & x \end{vmatrix}$, тогда так же $A = x \cdot (-1)^{2} \cdot \begin{vmatrix} x & y & 0 \\ 0 & x & y \\ 0 & 0 & x \end{vmatrix} + y \cdot (-1)^3 \cdot \begin{vmatrix} 0 & y & 0 \\ 0 & x & y \\ y & 0 & x \end{vmatrix} = x \cdot (x^3) - y \cdot (y^3) = x^4 - y^4$
    \\
    \\
    \\ \textbf{Ответ: } $x^4 - y^4$
    \\
    \\
    \\ \textbf{№ 9} Вычислите определитель, разложив его в произведение определителей
    \\
    \[
        \begin{vmatrix}\cos (\alpha_1 - \beta_1) & \cos (\alpha_1 - \beta_2) & ... & \cos (\alpha_1 - \beta_n) \\ \cos (\alpha_2 - \beta_1) & \cos (\alpha_2 - \beta_2) & ... & \cos (\alpha_2 - \beta_n) \\ ... & ... & ... & ... \\ \cos (\alpha_n - \beta_1) & \cos (\alpha_n - \beta_2) & ... & \cos (\alpha_n - \beta_n)\end{vmatrix}
    \]
    \[
        \begin{pmatrix}\cos (\alpha_1 - \beta_1) & \cos (\alpha_1 - \beta_2) & ... & \cos (\alpha_1 - \beta_n) \\ \cos (\alpha_2 - \beta_1) & \cos (\alpha_2 - \beta_2) & ... & \cos (\alpha_2 - \beta_n) \\ ... & ... & ... & ... \\ \cos (\alpha_n - \beta_1) & \cos (\alpha_n - \beta_2) & ... & \cos (\alpha_n - \beta_n)\end{pmatrix} = 
    \]
    \[
        = \begin{pmatrix}\cos \alpha_1 \cos \beta_1 + \sin \alpha_1 \sin \beta_1 & \cos \alpha_1 \cos \beta_2 + \sin \alpha_1 \sin \beta_2 & ... & \cos \alpha_1 \cos \beta_n + \sin \alpha_1 \sin \beta_n \\ \cos \alpha_2 \cos \beta_1 + \sin \alpha_2 \sin \beta_1 & \cos \alpha_2 \cos \beta_2 + \sin \alpha_2 \sin \beta_2 & ... & \cos \alpha_2 \cos \beta_n + \sin \alpha_2 \sin \beta_n \\ ... & ... & ... & ... \\ \cos \alpha_n \cos \beta_1 + \sin \alpha_n \sin \beta_1 & \cos \alpha_n \cos \beta_2 + \sin \alpha_n \sin \beta_2 & ... & \cos \alpha_n \cos \beta_n + \sin \alpha_n \sin \beta_n \end{pmatrix} = 
    \]
    \[
        \begin{pmatrix}\cos \alpha_1 & 0 & ... & 0 & \sin \alpha_1 \\ \cos \alpha_2 & 0 & ... & 0 & \sin \alpha_2 \\ ... & ... & ... & ... \\ \cos \alpha_n & 0 & ... & 0 & \sin \alpha_n \end{pmatrix} \cdot \begin{pmatrix}\cos \beta_1 & \cos \beta_2 & ... & \cos \beta_n \\ 0 & 0 & ... & 0 \\ ... & ... & ... & ... \\ 0 & 0 & ... & 0\\ \sin \beta_1 & \sin \beta_2 & ... & \sin \beta_n \end{pmatrix}
    \]
    \\ (Обе матрицы выше имеют размерность n на n)
    \\
    \\ Но так как в формуле определителя в каждом слагаемом встречается по одному элементу из каждой строки и каждого столбца матрицы, следовательно (так как в каждой из матриц есть нулевые строки или столбцы), определители обеих матриц из разложения равны 0, следовательно определитель главной матрицы равен 0 (так как определитель произведения матриц равен произведению определителей).
    \[
        \begin{vmatrix}\cos (\alpha_1 - \beta_1) & \cos (\alpha_1 - \beta_2) & ... & \cos (\alpha_1 - \beta_n) \\ \cos (\alpha_2 - \beta_1) & \cos (\alpha_2 - \beta_2) & ... & \cos (\alpha_2 - \beta_n) \\ ... & ... & ... & ... \\ \cos (\alpha_n - \beta_1) & \cos (\alpha_n - \beta_2) & ... & \cos (\alpha_n - \beta_n)\end{vmatrix} = 0
    \]
    \\
    \\
    \\ \textbf{Ответ:} 0
    \\
    \\
    \\ \textbf{№ 10} Найти наибольшее значение определителя порядка 3 при условии того, что все элементы равны 0 или 1.
    \\
    \\ Определитель матрицы А можно расписать по формуле:
    \\ $A = a_{11}a_{22}a_{33} + a_{21}a_{13}a_{32} + a_{12}a_{31}a_{23} - a_{13}a_{22}a_{31} - a_{12}a_{21}a_{33} - a_{11}a_{32}a_{23}$
    \\
    \\ Заметим, что каждая тройка пересекается с тремя другими, в одной точке с каждой. То есть, если мы хотя бы в одну точку тройки с минусом поставим 0, то занулится одна из троек с плюсом. Это значит, что мы не сможем получить определитель, равный 3м.
    \\
    \\ А вот определитель, равный 2 мы можем получить, для этого построим такую марицу $\begin{pmatrix}1 & 1 & 0 \\ 0 & 1 & 1 \\ 1 & 0 & 1\end{pmatrix}$.
    \\
    \\ $\begin{vmatrix}1 & 1 & 0 \\ 0 & 1 & 1 \\ 1 & 0 & 1\end{vmatrix} = 1 \cdot 1 \cdot 1 + 0 \cdot 0 \cdot 0 + 1 \cdot 1 \cdot 1 - 0 \cdot 1 \cdot 1 - 1 \cdot 0 \cdot 1 - 0 \cdot 1 \cdot 1 = 2$.
    \\
    \\ \textbf{Ответ: } 2
\end{document}