 % Всякое, чтобы работало - все библиотеки
 \documentclass[a4paper, 12pt]{article}
 
 \usepackage[T2A]{fontenc}
 \usepackage[russian, english]{babel}
 \usepackage[utf8]{inputenc}
 \usepackage{subfiles}
 \usepackage{ucs}
 \usepackage{textcomp}
 \usepackage{array}
 \usepackage{indentfirst}
 \usepackage{amsmath}
 \usepackage{amssymb}
 \usepackage{enumerate}
 \usepackage[margin=1.5cm]{geometry}
 \usepackage{authblk}
 \usepackage{tikz}
 \usepackage{icomma}
 \usepackage{gensymb}
 \usepackage{nicematrix, tikz}
  
 % Всякие мат штуки дополнительные
  
 \newcommand{\F}{\mathbb{F}}
 \newcommand{\di}{\frac}
 \renewcommand{\C}{\mathbb{C}}
 \newcommand{\N} {\mathbb{N}}
 \newcommand{\Z} {\mathbb{Z}}
 \newcommand{\R} {\mathbb{R}}
 \newcommand{\Q}{\mathbb{Q}}
 \newcommand{\ord} {\mathop{\rm ord}}
 \newcommand{\Ima}{\mathop{\rm Im}}
 \newcommand{\Rea}{\mathop{\rm Re}}
 \newcommand{\rk}{\mathop{\rm rk}}
 \newcommand{\arccosh}{\mathop{\rm arccosh}}
 \newcommand{\lker}{\mathop{\rm lker}}
 \newcommand{\rker}{\mathop{\rm rker}}
 \newcommand{\tr}{\mathop{\rm tr}}
 \newcommand{\St}{\mathop{\rm St}}
 \newcommand{\Mat}{\mathop{\rm Mat}}
 \newcommand{\grad}{\mathop{\rm grad}}
 \DeclareMathOperator{\spec}{spec}
 \renewcommand{\baselinestretch}{1.5}
 \everymath{\displaystyle}
  
 % Всякое для ускорения
 \renewcommand{\r}{\right}
 \renewcommand{\l}{\left}
 \newcommand{\Lra}{\Leftrightarrow}
 \newcommand{\ra}{\rightarrow}
 \newcommand{\la}{\leftarrow}
 \newcommand{\sm}{\setminus}
 \newcommand{\lm}{\lambda}
 \newcommand{\Sum}[2]{\overset{#2}{\underset{#1}{\sum}}}
 \newcommand{\Lim}[2]{\lim\limits_{#1 \rightarrow #2}}
 \newcommand{\p}[2]{\frac{\partial #1}{\partial #2}}
  
 % Заголовки
 \newcommand{\task}[1] {\noindent \textbf{Задача #1.} \hfill}
 \newcommand{\note}[1] {\noindent \textbf{Примечание #1.} \hfill}
  
 % Пространтсва для задач
 \newenvironment{proof}[1][Доказательство]{%
 \begin{trivlist}
     \item[\hskip \labelsep {\bfseries #1:}]
     \item \hspace{15pt}
     }{
     $ \hfill\blacksquare $
 \end{trivlist}
 \hfill\break
 }
 \newenvironment{solution}[1][Решение]{%
 \begin{trivlist}
     \item[\hskip \labelsep {\bfseries #1:}]
     \item \hspace{15pt}
     }{
 \end{trivlist}
 }
  
 \newenvironment{answer}[1][Ответ]{%
 \begin{trivlist}
     \item[\hskip \labelsep {\bfseries #1:}] \hskip \labelsep
     }{
 \end{trivlist}
 \hfill
 }
 \title{Дз по линейной алгебре 2 Смирнов Тимофей 236 ПМИ}
 \author{Тимофей Смирнов}
 \date{September 2023}
 
 \begin{document}
    {\center \bf \large ИДЗ по линейной алгебре 3 Смирнов Тимофей 236 ПМИ}
    \\
    \\ \textbf{№ 1} Найдите матрицу, обратную к данной матрице А:
    \[
       \begin{pNiceArray}{cccc}[last-col,margin]
           -3 & -4 & -4 & -2 & \\
           1 & 5 & 4 & 7 & \\
           -2 & -6 & -5 & -8 & \\
           -1 & -1 & -1 & 0 & \\
       \end{pNiceArray}
    \]
    \\
    \\ Чтобы найти обратную матрицу, запишем матрицу А и матрицу Е так (A|E). Затем проведем элементарные преобразования, пока слева не появится единичная марица, тогда справа будет наша матрица U, при умножении на которую матрица А будет превращаться в единичную.
    \\
    \[
       \begin{pNiceArray}{cccc|cccc}[last-col,margin]
           -3 & -4 & -4 & -2 & 1 & 0 & 0 & 0 & \\
           1 & 5 & 4 & 7 & 0 & 1 & 0 & 0 & \\
           -2 & -6 & -5 & -8 & 0 & 0 & 1 & 0 & \\
           -1 & -1 & -1 & 0 & 0 & 0 & 0 & 1 & \\
       \end{pNiceArray}
    \]
    \\
    \\ Поменяем местами первую и вторую строки ---------------------------------------------------------------------------------
    \\
    \[
       \begin{pNiceArray}{cccc|cccc}[last-col,margin]
            1 & 5 & 4 & 7 & 0 & 1 & 0 & 0 & \\
           -3 & -4 & -4 & -2 & 1 & 0 & 0 & 0 & +3(1)\\
           -2 & -6 & -5 & -8 & 0 & 0 & 1 & 0 & +2(1)\\
           -1 & -1 & -1 & 0 & 0 & 0 & 0 & 1 & +1(1)\\
       \end{pNiceArray}
       \to
       \begin{pNiceArray}{cccc|cccc}[last-col,margin]
            1 & 5 & 4 & 7 & 0 & 1 & 0 & 0 & \\
            0 & 11 & 8 & 19 & 1 & 3 & 0 & 0 & :11\\
            0 & 4 & 3 & 6 & 0 & 2 & 1 & 0 & \\
            0 & 4 & 3 & 7 & 0 & 1 & 0 & 1 & \\
        \end{pNiceArray}
        \to
    \]
    \\
    \[
       \begin{pNiceArray}{cccc|cccc}[last-col,margin]
            1 & 5 & 4 & 7 & 0 & 1 & 0 & 0 & \\
            0 & 1 & \frac{8}{11} & \frac{19}{11} & \frac{1}{11} & \frac{3}{11} & 0 & 0 & \\
            0 & 4 & 3 & 6 & 0 & 2 & 1 & 0 & -4(2) \\
            0 & 4 & 3 & 7 & 0 & 1 & 0 & 1 & -4(2) \\
        \end{pNiceArray}
        \to
        \begin{pNiceArray}{cccc|cccc}[last-col,margin]
            1 & 5 & 4 & 7 & 0 & 1 & 0 & 0 & \\
            0 & 1 & \frac{8}{11} & \frac{19}{11} & \frac{1}{11} & \frac{3}{11} & 0 & 0 & \\
            0 & 0 & \frac{1}{11} & -\frac{10}{11} & -\frac{4}{11} & \frac{10}{11} & 1 & 0 & \times 11 \\
            0 & 0 & \frac{1}{11} & \frac{1}{11} & -\frac{4}{11} & -\frac{1}{11} & 0 & 1 & \times 11 \\
        \end{pNiceArray}
    \]
    \\
    \[
        \begin{pNiceArray}{cccc|cccc}[last-col,margin]
            1 & 5 & 4 & 7 & 0 & 1 & 0 & 0 & \\
            0 & 1 & \frac{8}{11} & \frac{19}{11} & \frac{1}{11} & \frac{3}{11} & 0 & 0 & \\
            0 & 0 & 1 & -10 & -4 & 10 & 11 & 0 & \\
            0 & 0 & 1 & 1 & -4 & -1 & 0 & 11 & -1(3) \\
        \end{pNiceArray}
        \to
        \begin{pNiceArray}{cccc|cccc}[last-col,margin]
            1 & 5 & 4 & 7 & 0 & 1 & 0 & 0 & \\
            0 & 1 & \frac{8}{11} & \frac{19}{11} & \frac{1}{11} & \frac{3}{11} & 0 & 0 & \\
            0 & 0 & 1 & -10 & -4 & 10 & 11 & 0 & \\
            0 & 0 & 0 & 11 & 0 & -11 & -11 & 11 & :11\\
        \end{pNiceArray}
    \]
    \\
    \[
        \begin{pNiceArray}{cccc|cccc}[last-col,margin]
            1 & 5 & 4 & 7 & 0 & 1 & 0 & 0 & \\
            0 & 1 & \frac{8}{11} & \frac{19}{11} & \frac{1}{11} & \frac{3}{11} & 0 & 0 & \\
            0 & 0 & 1 & -10 & -4 & 10 & 11 & 0 & +10(4)\\
            0 & 0 & 0 & 1 & 0 & -1 & -1 & 1 & \\
        \end{pNiceArray}
        \to
        \begin{pNiceArray}{cccc|cccc}[last-col,margin]
            1 & 5 & 4 & 7 & 0 & 1 & 0 & 0 & \\
            0 & 1 & \frac{8}{11} & \frac{19}{11} & \frac{1}{11} & \frac{3}{11} & 0 & 0 & -\frac{19}{11}(4)\\
            0 & 0 & 1 & 0 & -4 & 0 & 1 & 10 & \\
            0 & 0 & 0 & 1 & 0 & -1 & -1 & 1 & \\
        \end{pNiceArray}
    \]
    \[
        \begin{pNiceArray}{cccc|cccc}[last-col,margin]
            1 & 5 & 4 & 7 & 0 & 1 & 0 & 0 & \\
            0 & 1 & \frac{8}{11} & 0 & \frac{1}{11} & 2 & \frac{19}{11} & -\frac{19}{11} & -\frac{8}{11}(4)\\
            0 & 0 & 1 & 0 & -4 & 0 & 1 & 10 & \\
            0 & 0 & 0 & 1 & 0 & -1 & -1 & 1 & \\
        \end{pNiceArray}
        \to
        \begin{pNiceArray}{cccc|cccc}[last-col,margin]
            1 & 5 & 4 & 7 & 0 & 1 & 0 & 0 & -7(4)\\
            0 & 1 & 0 & 0 & 3 & 2 & 1 & -9 & \\
            0 & 0 & 1 & 0 & -4 & 0 & 1 & 10 & \\
            0 & 0 & 0 & 1 & 0 & -1 & -1 & 1 & \\
        \end{pNiceArray}
    \]
    \\
    \[
        \begin{pNiceArray}{cccc|cccc}[last-col,margin]
            1 & 5 & 4 & 0 & 0 & 8 & 7 & -7 & -4(3)\\
            0 & 1 & 0 & 0 & 3 & 2 & 1 & -9 & \\
            0 & 0 & 1 & 0 & -4 & 0 & 1 & 10 & \\
            0 & 0 & 0 & 1 & 0 & -1 & -1 & 1 & \\
        \end{pNiceArray}
        \to
        \begin{pNiceArray}{cccc|cccc}[last-col,margin]
            1 & 5 & 0 & 0 & 16 & 8 & 3 & -47 & -5(4)\\
            0 & 1 & 0 & 0 & 3 & 2 & 1 & -9 & \\
            0 & 0 & 1 & 0 & -4 & 0 & 1 & 10 & \\
            0 & 0 & 0 & 1 & 0 & -1 & -1 & 1 & \\
        \end{pNiceArray}
    \]
    \\
    \[
        \begin{pNiceArray}{cccc|cccc}[last-col,margin]
            1 & 0 & 0 & 0 & 1 & -2 & -2 & -2 & \\
            0 & 1 & 0 & 0 & 3 & 2 & 1 & -9 & \\
            0 & 0 & 1 & 0 & -4 & 0 & 1 & 10 & \\
            0 & 0 & 0 & 1 & 0 & -1 & -1 & 1 & \\
        \end{pNiceArray}
    \]
    \\
    \\ Вот мы и получили обратную к А матрицу, проверим:
    \\
    \[ 
        \begin{pmatrix}
            -3 & -4 & -4 & -2 \\
           1 & 5 & 4 & 7 \\
           -2 & -6 & -5 & -8 \\
           -1 & -1 & -1 & 0 \\
        \end{pmatrix} \cdot 
        \begin{pmatrix}
            1 & -2 & -2 & -2 \\
            3 & 2 & 1 & -9 \\
            -4 & 0 & 1 & 10 \\
            0 & -1 & -1 & 1 \\
        \end{pmatrix} = \begin{pmatrix}1 & 0 & 0 & 0\\0 & 1 & 0 & 0\\0 & 0 & 1 & 0\\0 & 0 & 0 & 1\end{pmatrix}
    \]
    \\
    \par \ \ \ \ \textbf{Ответ: } $A^{-1} = \begin{pmatrix}
        1 & -2 & -2 & -2 \\
        3 & 2 & 1 & -9 \\
        -4 & 0 & 1 & 10 \\
        0 & -1 & -1 & 1 \\
    \end{pmatrix}$
    \\
    \\
    \\
    \\ \textbf{№ 2} Решите уравнение относительно неизвестной перестановки X:
    \[
        \left( \begin{pmatrix}1 & 2 & 3 & 4 & 5 & 6 & 7 & 8 \\ 8 & 6 & 4 & 2 & 7 & 1 & 5 & 3 \end{pmatrix}^{12} \cdot \begin{pmatrix}1 & 2 & 3 & 4 & 5 & 6 & 7 & 8 \\ 2 & 3 & 8 & 4 & 7 & 5 & 1 & 6 \end{pmatrix}^{-1} \right)^{112} \cdot X = \begin{pmatrix}1 & 2 & 3 & 4 & 5 & 6 & 7 & 8 \\ 2 & 6 & 7 & 1 & 8 & 4 & 5 & 3 \end{pmatrix}
    \]
    \\
    \\ 1). $\begin{pmatrix}1 & 2 & 3 & 4 & 5 & 6 & 7 & 8 \\ 8 & 6 & 4 & 2 & 7 & 1 & 5 & 3 \end{pmatrix} = (183426)(57)$ 
    \\ 
    \par НОК длин циклов этой перестановки перестановки равен 6. 
    \par 12 делится на 6 нацело, следовательно $\begin{pmatrix}1 & 2 & 3 & 4 & 5 & 6 & 7 & 8 \\ 8 & 6 & 4 & 2 & 7 & 1 & 5 & 3 \end{pmatrix}^{12} = \begin{pmatrix}1 & 2 & 3 & 4 & 5 & 6 & 7 & 8 \\ 8 & 6 & 4 & 2 & 7 & 1 & 5 & 3 \end{pmatrix}$
    \\
    \\
    \\ 2). $\begin{pmatrix}1 & 2 & 3 & 4 & 5 & 6 & 7 & 8 \\ 2 & 3 & 8 & 4 & 7 & 5 & 1 & 6 \end{pmatrix}^{-1} = \begin{pmatrix}1 & 2 & 3 & 4 & 5 & 6 & 7 & 8 \\ 7 & 1 & 2 & 4 & 6 & 8 & 5 & 3 \end{pmatrix}$
    \\
    \\
    \\ 3). $\begin{pmatrix}1 & 2 & 3 & 4 & 5 & 6 & 7 & 8 \\ 8 & 6 & 4 & 2 & 7 & 1 & 5 & 3 \end{pmatrix}^{12} \cdot \begin{pmatrix}1 & 2 & 3 & 4 & 5 & 6 & 7 & 8 \\ 2 & 3 & 8 & 4 & 7 & 5 & 1 & 6 \end{pmatrix}^{-1} \\ = \begin{pmatrix}1 & 2 & 3 & 4 & 5 & 6 & 7 & 8 \\ 8 & 6 & 4 & 2 & 7 & 1 & 5 & 3 \end{pmatrix} \cdot \begin{pmatrix}1 & 2 & 3 & 4 & 5 & 6 & 7 & 8 \\ 2 & 3 & 8 & 4 & 7 & 5 & 1 & 6 \end{pmatrix} = \begin{pmatrix}1 & 2 & 3 & 4 & 5 & 6 & 7 & 8 \\ 6 & 4 & 3 & 2 & 5 & 7 & 8 & 1 \end{pmatrix}$
    \\
    \\ 4). $\begin{pmatrix}1 & 2 & 3 & 4 & 5 & 6 & 7 & 8 \\ 6 & 4 & 3 & 2 & 5 & 7 & 8 & 1 \end{pmatrix} = (1678)(24)(3)(5)$
    \\
    \par НОК длин циклов перестановки равен 4. 112 делится на 4 нацело, следовательно 
    \par $\begin{pmatrix}1 & 2 & 3 & 4 & 5 & 6 & 7 & 8 \\ 6 & 4 & 3 & 2 & 5 & 7 & 8 & 1 \end{pmatrix}^{112} = \begin{pmatrix}1 & 2 & 3 & 4 & 5 & 6 & 7 & 8 \\ 6 & 4 & 3 & 2 & 5 & 7 & 8 & 1 \end{pmatrix}$
    \\
    \\
    \\ 5). Получаем уравнение $\begin{pmatrix}1 & 2 & 3 & 4 & 5 & 6 & 7 & 8 \\ 6 & 4 & 3 & 2 & 5 & 7 & 8 & 1 \end{pmatrix} \cdot X = \begin{pmatrix}1 & 2 & 3 & 4 & 5 & 6 & 7 & 8 \\ 2 & 6 & 7 & 1 & 8 & 4 & 5 & 3 \end{pmatrix}$
    \\
    \\ Получаем $X = \begin{pmatrix}1 & 2 & 3 & 4 & 5 & 6 & 7 & 8 \\ 4 & 1 & 6 & 8 & 7 & 2 & 5 & 3 \end{pmatrix}$
    \\
    \\ \textbf{Ответ: } $X = \begin{pmatrix}1 & 2 & 3 & 4 & 5 & 6 & 7 & 8 \\ 4 & 1 & 6 & 8 & 7 & 2 & 5 & 3 \end{pmatrix}$
    \\
    \\ \textbf{№ 3} Определите чётность перестановки
    \\
    \[ 
        \begin{pmatrix}1 & 2 & \dots & 97 & 98 & \dots & 327 & 328 & \dots & 470 \\ 374 & 375 & \dots & 470 & 144 & \dots & 373 & 1 & \dots & 143\end{pmatrix}
    \]
    \\
    \\ 1). Для каждого из чисел от 374 до 470 мы имеем инверсии с числами от 1 до 373. Всего таких перестановок будет $(470 - 374 + 1) \cdot 373 = 36181$
    \\
    \\ 2). Для каждого из чисел от 144 до 373 мы имеем инверсии с числами от 1 до 143 (инверсии с числами от 374 до 470 мы уже посчитали). Всего таких перестановок будет $(373 - 144 + 1) \cdot 143 = 32890$
    \\
    \\ 3). Число $36181 + 32890 = 69071$ нечетно, следовательно перестановка нечетна.
    \\
    \par \ \ \ \ \textbf{Ответ: } Перестановка нечетна.
    \\
    \\
    \\ \textbf{№ 4} Вычислите определитель матрицы A
    \[
        \begin{vmatrix}
            0 & 0 & x & 0 & 0 & 8 \\
            0 & 0 & 7 & 6 & 0 & 5 \\
            0 & 3 & 0 & 3 & 7 & 5 \\
            1 & 2 & x & 0 & x & 7 \\
            x & 1 & 3 & x & x & 6 \\
            0 & 4 & 8 & 6 & 9 & 5 \\
        \end{vmatrix}
    \]
    \\
    \\ Разложим определитель матрицы по первой строке:
    \\
    \\ $\det A = 0 \cdot A_{11} + 0 \cdot A_{12} + x \cdot A_{13} + 0 \cdot A_{14} + 0 \cdot A_{15} + 8 \cdot A_{16} = x \cdot A_{13} + 8 \cdot A_{16}$
    \\
    \\ 1). $x \cdot A_{13} = x \cdot M = x \cdot (-1)^{4} \cdot \begin{vmatrix}
        0 & 0 & 6 & 0 & 5 \\
        0 & 3 & 3 & 7 & 5 \\
        1 & 2 & 0 & x & 7 \\
        x & 1 & x & x & 6 \\
        0 & 4 & 6 & 9 & 5 \\
    \end{vmatrix} = x \cdot \left(6 \cdot M_{13} + 5 \cdot M_{15}\right)$
    \\
    \[
        M_{13} = (-1)^{4} \cdot \begin{vmatrix}
            0 & 3 & 7 & 5 \\
            1 & 2 & x & 7 \\
            x & 1 & x & 6 \\
            0 & 4 & 9 & 5 \\
        \end{vmatrix} = 3 \cdot (-1)^3 \cdot \begin{vmatrix}
            1 & x & 7 \\
            x & x & 6 \\
            0 & 9 & 5 \\
        \end{vmatrix} + 7 \cdot (-1)^4 \cdot \begin{vmatrix}
            1 & 2 & 7 \\
            x & 1 & 6 \\
            0 & 4 & 5 \\
        \end{vmatrix} + 5 \cdot (-1)^5 \cdot \begin{vmatrix}
            1 & 2 & x \\
            x & 1 & x \\
            0 & 4 & 9 \\
        \end{vmatrix} =
    \]
    \\ Посчитаем определители матриц 3 на 3 мы можем просто по правилу "снежинки", что делается вполне устно.
    \[
        (-3) \cdot (- 5 x^{2} + 68 x - 54) + 7 \cdot (18 x - 19) + (-5) \cdot (4 x^{2} - 22 x + 9) =    
        - 5 x^{2} + 32 x - 16    
    \]
    \\
    \[
        M_{15} = (-1)^6 \cdot \begin{vmatrix}
            0 & 3 & 3 & 7 \\
            1 & 2 & 0 & x \\
            x & 1 & x & x \\
            0 & 4 & 6 & 9 \\
        \end{vmatrix} = 3 \cdot (-1)^3 \cdot  \begin{vmatrix}
            1 & 0 & x \\
            x & x & x \\
            0 & 6 & 9 \\
        \end{vmatrix} + 3 \cdot (-1)^4 \cdot \begin{vmatrix}
            1 & 2 & x \\
            x & 1 & x \\
            0 & 4 & 9 \\
        \end{vmatrix}  + 7 \cdot (-1)^5 \cdot \begin{vmatrix}
            1 & 2 & 0 \\
            x & 1 & x \\
            0 & 4 & 6 \\
        \end{vmatrix} = 
    \]
    \[
        (-3) \cdot (3 x (2 x + 1)) + 3 \cdot (4 x^{2} - 22 x + 9) + (-7) \cdot (6 - 16 x)  = - 6 x^{2} + 37 x - 15 
    \]
    \\ --------------------------------------------------------------------------------------------------------------------------------------------------------------------
    \[
        x \cdot A_{13} = x \cdot (6 \cdot (- 5 x^{2} + 32 x - 16) + 5 \cdot ( - 6 x^{2} + 37 x - 15))  = -60x^{3} + 377x^2 - 171x  
    \]
    \\ --------------------------------------------------------------------------------------------------------------------------------------------------------------------
    \\
    \\
    \\ 2). $8 \cdot A_{16} = 8 \cdot (-1)^{7} \cdot H = 8 \cdot (-1)^{7} \cdot \begin{vmatrix}
        0 & 0 & 7 & 6 & 0 \\
        0 & 3 & 0 & 3 & 7 \\
        1 & 2 & x & 0 & x \\
        x & 1 & 3 & x & x \\
        0 & 4 & 8 & 6 & 9 \\
    \end{vmatrix} = -8 \cdot (7 \cdot H_{13} + 6 \cdot H_{14})$
    \\
    \[
        H_{13} = (-1)^4 \cdot \begin{vmatrix}
            0 & 3 & 3 & 7 \\
            1 & 2 & 0 & x \\
            x & 1 & x & x \\
            0 & 4 & 6 & 9 \\
        \end{vmatrix} = 3 \cdot (-1)^3 \cdot \begin{vmatrix}
            1 & 0 & x \\
            x & x & x \\
            0 & 6 & 9 \\
        \end{vmatrix} + 3 \cdot (-1)^4 \cdot \begin{vmatrix}
            1 & 2 & x \\
            x & 1 & x \\
            0 & 4 & 9 \\
        \end{vmatrix} + 7 \cdot (-1)^5 \cdot \begin{vmatrix}
            1 & 2 & 0 \\
            x & 1 & x \\
            0 & 4 & 6 \\
        \end{vmatrix} = 
    \]
    \[
        (-3) \cdot (6x^2 + 3x) + 3 \cdot (4 x^{2} - 22 x + 9) - 7 \cdot (6 - 16 x) = - 6 x^{2} + 37 x - 15
    \]
    \\
    \[
        H_{14} = (-1)^5 \cdot \begin{vmatrix}
            0 & 3 & 0 & 7 \\
            1 & 2 & x & x \\
            x & 1 & 3 & x \\
            0 & 4 & 8 & 9 \\
        \end{vmatrix} = (-1) \cdot (3 \cdot (-1)^3 \cdot \begin{vmatrix}
            1 & x & x \\
            x & 3 & x \\
            0 & 8 & 9 \\
        \end{vmatrix} + 7 \cdot (-1)^5 \cdot \begin{vmatrix}
            1 & 2 & x \\
            x & 1 & 3 \\
            0 & 4 & 8 \\
        \end{vmatrix}) =    
    \]
    \[
        (-1) \cdot (-3 \cdot (- x^{2} - 8 x + 27) - 7 \cdot (4 x^{2} - 16 x - 4)) = 25 x^{2} - 136 x + 53
    \]
    \\ --------------------------------------------------------------------------------------------------------------------------------------------------------------------
    \[
        8 \cdot A_{16} = -8 \cdot (7 \cdot (- 6 x^{2} + 37 x - 15) + 6 \cdot (25 x^{2} - 136 x + 53)) = - 864 x^{2} + 4456 x - 1704
    \]
    \\ --------------------------------------------------------------------------------------------------------------------------------------------------------------------
    \\
    \\ Наконец-то посчитаем $\det A = x \cdot A_{13} + 8 \cdot A_{16} = -60x^{3} + 377x^2 - 171x - 864 x^{2} + 4456 x - 1704 = \ \ \ = - 60 x^{3} - 487 x^{2} + 4285 x - 1704$
    \\
    \\
    \\ \textbf{Ответ: } $\det A = - 60 x^{3} - 487 x^{2} + 4285 x - 1704$
    \\
    \\ \textbf{№ 5} Найдите коэффициент при $x^5$ в выражении определителя
    \\
    \[
        \begin{vmatrix}
            3 & 9 & x & 1 & 7 & 4 & 3 \\
            9 & x & 8 & 2 & 8 & 4 & 2 \\
            x & 8 & 3 & 10 & 1 & 8 & x \\
            1 & 2 & 10 & 5 & 3 & x & 7 \\
            7 & 8 & 1 & 3 & x & 10 & 5 \\
            4 & 4 & 8 & x & 10 & 9 & 6 \\
            3 & 2 & x & 7 & 5 & 6 & 10 \\
        \end{vmatrix}
    \]
    \\
    \\ Если к строке или столбцу матрицы прибавить другой столбец или строку той же матрицы, умноженный на скаляр, то определитель не изменится.
    \\
    \\ Добавим к 7му столбцу нашей стрицы первый столбец, множенный на -1 и получим это (этот определитель равен предыдущем по свойству определителей):
    \\
    \[
        \begin{vmatrix}
            3 & 9 & x & 1 & 7 & 4 & 0 \\
            9 & x & 8 & 2 & 8 & 4 & -7 \\
            x & 8 & 3 & 10 & 1 & 8 & 0 \\
            1 & 2 & 10 & 5 & 3 & x & 6 \\
            7 & 8 & 1 & 3 & x & 10 & -2 \\
            4 & 4 & 8 & x & 10 & 9 & 2 \\
            3 & 2 & x & 7 & 5 & 6 & 7 \\
        \end{vmatrix}
    \]
    \\
    \\ Теперь к 7й строке матрицы прибавим первую, тоже умноженную на -1:
    \\
    \[
        \begin{vmatrix}
            3 & 9 & x & 1 & 7 & 4 & 0 \\
            9 & x & 8 & 2 & 8 & 4 & -7 \\
            x & 8 & 3 & 10 & 1 & 8 & 0 \\
            1 & 2 & 10 & 5 & 3 & x & 6 \\
            7 & 8 & 1 & 3 & x & 10 & -2 \\
            4 & 4 & 8 & x & 10 & 9 & 2 \\
            0 & -7 & 0 & 6 & -2 & 2 & 7 \\
        \end{vmatrix}
    \]
    \\
    \\ У днанной матрицы тоже тот же самый определитель, а, следовательно, и коэффициент при $x^5$ у него тот же самый.
    \\
    \\ В каждом слагаемом в формуле определителя мы берем ровно по одному элементу из каждой строки и столбца.  У нас осталось всего 6 иксов, тогда $x^5$ получается выбором 5 из них (или НЕ выбором одного из 6), то есть 6ю способами. Рассмотрим эти способы и просуммируем их:
    \\
    \par \ \ \ \ 1). Пусть мы не возьмем икс из первой строки, тогда мы получим такую формулу слагаемого: $sgn (S_1) \cdot 0 \cdot x \cdot x \cdot x \cdot x \cdot x \cdot 0 = 0;$ ($S_1$ здесь это перестановка, которой соответствует такой выбор в данной ситуации она такова: $\begin{pmatrix}1 & 2 & 3 & 4 & 5 & 6 & 7 \\ 7 & 2 & 1 & 6 & 5 & 4 & 3 \end{pmatrix}$).
    \\
    \par \ \ \ \ 2). Пусть мы не возьмем икс из 2й строки, $S_2 = \begin{pmatrix}1 & 2 & 3 & 4 & 5 & 6 & 7 \\ 3 & 7 & 1 & 6 & 5 & 4 & 2 \end{pmatrix}$, наше слагаемое примет такой вид: $sgn (S_2) \cdot x \cdot (-7) \cdot x \cdot x \cdot x \cdot x \cdot (-7); sgn(S_2) = -1 \ \Rightarrow \ $ (так как в перестановке 13 инверсий) получаем $ (-1) \cdot x^5 \cdot 49 = -49x^5$
    \\
    \par \ \ \ \ 3). Пусть мы не возьмем икс из 3й строки, тогда $S_3 = \begin{pmatrix}1 & 2 & 3 & 4 & 5 & 6 & 7 \\ 3 & 2 & 7 & 6 & 5 & 4 & 1 \end{pmatrix}$, в этой перестановке 10 инверсий, но нам это не важно, так как из 3й строки будет взят 0, который занулит нам все слагаемое.
    \\
    \par \ \ \ \ 4). Пусть мы не возьмем икс из 4й строки, тогда $S_4 = \begin{pmatrix}1 & 2 & 3 & 4 & 5 & 6 & 7 \\ 3 & 2 & 1 & 7 & 5 & 4 & 6 \end{pmatrix}$, в этой перестановке 7 инверсий $\ \Rightarrow \ sgn (S_4) = -1 \ \Rightarrow \ sgn(S_4) \cdot x^3 \cdot 6 \cdot x^2 \cdot 2 = -12x^5$.
    \\
    \par \ \ \ \ 5). Пусть мы не возьмем икс из 5й строки, тогда $S_5 = \begin{pmatrix}1 & 2 & 3 & 4 & 5 & 6 & 7 \\ 3 & 2 & 1 & 6 & 7 & 4 & 5 \end{pmatrix}$, в этой перестановке 7 инверский $\ \Rightarrow \ sgb (S_5) = -1 \ \Rightarrow \ sgn(S_5) \cdot x^4 \cdot (-2) \cdot x \cdot (-2) = -4x^5$
    \\
    \par \ \ \ \ 6). Пусть мы не возьмем икс из 6й строки, тогда $S_6 = \begin{pmatrix}1 & 2 & 3 & 4 & 5 & 6 & 7 \\ 3 & 2 & 1 & 6 & 5 & 7 & 4 \end{pmatrix}$, в этой перестановке 7 инверсий $\ \Rightarrow \ sgn(S_6) = -1 \ \Rightarrow \ sgn(S_6) x^5 \cdot 2 \cdot 6 = -12x^5$
    \\
    \\ Сумма коэффициентов при $x^5$ равна $0 - 49 + 0 - 12 - 4 - 12 = -77$
    \\
    \par \ \ \ \ \textbf{Ответ: } -77
\end{document}