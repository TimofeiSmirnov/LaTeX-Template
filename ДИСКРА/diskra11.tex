 % Всякое, чтобы работало - все библиотеки
 \documentclass[a4paper, 12pt]{article}
 
 \usepackage[T2A]{fontenc}
 \usepackage[russian, english]{babel}
 \usepackage[utf8]{inputenc}
 \usepackage{subfiles}
 \usepackage{ucs}
 \usepackage{textcomp}
 \usepackage{array}
 \usepackage{indentfirst}
 \usepackage{amsmath}
 \usepackage{amssymb}
 \usepackage{enumerate}
 \usepackage[margin=1.5cm]{geometry}
 \usepackage{authblk}
 \usepackage{tikz}
 \usepackage{icomma}
 \usepackage{gensymb}
 \usepackage{nicematrix, tikz}
 \usepackage{tikz}
  
 % Всякие мат штуки дополнительные
  
 \newcommand{\F}{\mathbb{F}}
 \newcommand{\di}{\frac}
 \renewcommand{\C}{\mathbb{C}}
 \newcommand{\N} {\mathbb{N}}
 \newcommand{\Z} {\mathbb{Z}}
 \newcommand{\R} {\mathbb{R}}
 \newcommand{\Q}{\mathbb{Q}}
 \newcommand{\ord} {\mathop{\rm ord}}
 \newcommand{\Ima}{\mathop{\rm Im}}
 \newcommand{\Rea}{\mathop{\rm Re}}
 \newcommand{\rk}{\mathop{\rm rk}}
 \newcommand{\arccosh}{\mathop{\rm arccosh}}
 \newcommand{\lker}{\mathop{\rm lker}}
 \newcommand{\rker}{\mathop{\rm rker}}
 \newcommand{\tr}{\mathop{\rm tr}}
 \newcommand{\St}{\mathop{\rm St}}
 \newcommand{\Mat}{\mathop{\rm Mat}}
 \newcommand{\grad}{\mathop{\rm grad}}
 \DeclareMathOperator{\spec}{spec}
 \renewcommand{\baselinestretch}{1.5}
 \everymath{\displaystyle}
  
 % Всякое для ускорения
 \renewcommand{\r}{\right}
 \renewcommand{\l}{\left}
 \newcommand{\Lra}{\Leftrightarrow}
 \newcommand{\ra}{\rightarrow}
 \newcommand{\la}{\leftarrow}
 \newcommand{\sm}{\setminus}
 \newcommand{\lm}{\lambda}
 \newcommand{\Sum}[2]{\overset{#2}{\underset{#1}{\sum}}}
 \newcommand{\Lim}[2]{\lim\limits_{#1 \rightarrow #2}}
 \newcommand{\p}[2]{\frac{\partial #1}{\partial #2}}
  
 % Заголовки
 \newcommand{\task}[1] {\noindent \textbf{Задача #1.} \hfill}
 \newcommand{\note}[1] {\noindent \textbf{Примечание #1.} \hfill}
  
 % Пространтсва для задач
 \newenvironment{proof}[1][Доказательство]{%
 \begin{trivlist}
     \item[\hskip \labelsep {\bfseries #1:}]
     \item \hspace{15pt}
     }{
     $ \hfill\blacksquare $
 \end{trivlist}
 \hfill\break
 }
 \newenvironment{solution}[1][Решение]{%
 \begin{trivlist}
     \item[\hskip \labelsep {\bfseries #1:}]
     \item \hspace{15pt}
     }{
 \end{trivlist}
 }
  
 \newenvironment{answer}[1][Ответ]{%
 \begin{trivlist}
     \item[\hskip \labelsep {\bfseries #1:}] \hskip \labelsep
     }{
 \end{trivlist}
 \hfill
 }
 \title{Дз по линейной алгебре 2 Смирнов Тимофей 236 ПМИ}
 \author{Тимофей Смирнов}
 \date{September 2023}
 
 \begin{document}
    {\center \bf \large ДЗ по дискретной математике 11 Смирнов Тимофей 236 ПМИ}
    \\
    \\ \textbf{Д11.1} Сколько компонент связности в лесу на 6 вершинах с 4 рёбрами? Приведите пример такого леса.
    \\
    \\ \textbf{Решение: } Цикломатическое число леса равно 0. Так как количество вершин в нашем графе 6, а ребер 4, то чтобы цикломатрическое число было равно нулю в нем должно быть ровно 2 компоненты связности.
    \\
    \\ Пример такого графа:
    \\\\
    $\begin{tikzpicture}[node distance={15mm}, thick, main/.style = {draw, circle}] 
        \node[main] (1) {$1$}; 
        \node[main] (2) [above right of=1] {$2$}; 
        \node[main] (3) [above right of=2] {$3$}; 
        \node[main] (4) [right of=3] {$4$}; 
        \node[main] (5) [below right of=4] {$5$}; 
        \node[main] (6) [below right of=5] {$6$}; 
        \draw[-] (1) -- (2); 
        \draw[-] (3) -- (4); 
        \draw[-] (1) -- (5);
        \draw[-] (2) -- (3);
      \end{tikzpicture}$
    \\
    \\
    \\ \textbf{Д11.2} Сколько простых путей может быть в дереве на n вершинах? Укажите все возможные варианты.
    (Ответ, разумеется, должен быть обоснован.)
    \\
    \\ \textbf{Решение: } в дереве между любыми двумя вершинами существует единственный простой путь. Следовательно, в дереве столько же простых путей, сколько в нем упорядоченных пар вершин. То есть для дерева на n вершинах существует $n^2$ упорядоченных пар вершин (пары типа $(x, x)$ тоже считаются, так как они тоже образуют путь), а, следовательно, и простых путей.
    \\
    \\ \textbf{Ответ: } $n^2$
    \\
    \\ \textbf{Д11.3} Найдите наибольшее количество вершин в связном графе, сумма степеней вершин в котором равна 20.
    \\
    \\ \textbf{Решение: }
    \\ Заметим, что при добавлении новой вершины в связный граф, его суммарная степень вершин увеличивается как минимум на 2, так как добавляется 1 к степени новой вершины и 1 к степени вершины, к которой присоединяется новая. Если в графе 1 вершина, то суммарная степень вершин равна 0. Следовательно, чтобы получить суммарную степень вершин 20, мы можем добавить не более 10 ребер. Что в итоге даст 11.
    \\
    \\ Тогда приведем пример графа с 11 вершинами:
    \\
    \\
    $\begin{tikzpicture}[node distance={15mm}, thick, main/.style = {draw, circle}] 
        \node[main] (1) {$1$}; 
        \node[main] (2) [above right of=1] {$2$}; 
        \node[main] (3) [above right of=2] {$3$}; 
        \node[main] (4) [right of=3] {$4$}; 
        \node[main] (5) [below right of=4] {$5$}; 
        \node[main] (6) [below right of=5] {$6$}; 
        \node[main] (7) [below left of=6] {$7$}; 
        \node[main] (8) [below left of=7] {$8$}; 
        \node[main] (9) [left of=8] {$9$}; 
        \node[main] (10) [above left of=9] {$10$};
        \node[main] (11) [above left of=7] {$11$};
        \draw[-] (1) -- (2); 
        \draw[-] (2) -- (3);
        \draw[-] (3) -- (4);
        \draw[-] (4) -- (5);
        \draw[-] (5) -- (6);
        \draw[-] (6) -- (7);
        \draw[-] (7) -- (8);
        \draw[-] (8) -- (9);
        \draw[-] (10) -- (9);
        \draw[-] (10) -- (11);
      \end{tikzpicture}$
    \\
    \\
    \\ \textbf{Д11.4} а) Приведите пример дерева на 14 вершинах, в котором есть ровно две вершины степени 6 и нет
    ни одной вершины степени 2. б) В дереве на 13 вершинах есть ровно две вершины степени 6. Следует
    ли из этого, что в этом дереве есть вершина степени 2?
    \\
    \\ \textbf{Решение: }
    \\
    \\ a).
    \\
    \\
    $\begin{tikzpicture}[node distance={15mm}, thick, main/.style = {draw, circle}] 
        \node[main] (1) {$1$}; 
        \node[main] (3) [above of=1] {$3$}; 
        \node[main] (2) [right of=1] {$2$}; 
        \node[main] (7) [below  of=1] {$7$}; 
        \node[main] (6) [below left of=1] {$6$}; 
        \node[main] (5) [left of=1] {$5$}; 
        \node[main] (4) [above left of=1] {$4$}; 
        \node[main] (8) [above of=2] {$8$};
        \node[main] (9) [above right of=2] {$9$};
        \node[main] (10) [right of=2] {$10$};
        \node[main] (11) [below right of=2] {$11$};
        \node[main] (12) [below of=2] {$12$};
        \node[main] (13) [above right of=10] {$13$};
        \node[main] (14) [below right of=10] {$14$};
        \draw[-] (1) -- (2); 
        \draw[-] (1) -- (3);
        \draw[-] (1) -- (4);
        \draw[-] (1) -- (5);
        \draw[-] (1) -- (6);
        \draw[-] (1) -- (7);
        \draw[-] (2) -- (8);
        \draw[-] (2) -- (9);
        \draw[-] (2) -- (10);
        \draw[-] (2) -- (11);
        \draw[-] (2) -- (12);
        \draw[-] (10) -- (13);
        \draw[-] (10) -- (14);     
      \end{tikzpicture}$
    \\
    \\
    \\ б). Если в дереве 13 вершин, то в нем 13 - 1 = 12 ребер. Пусть $d_1, ..., d_{13}$ - это степени вершин, тогда:
    \[ d_1 + ... + d_{13} = 2 \cdot 12 = 24 \]
    \\ Но мы знаем, что две степени вершин равны 6, пусть это $d_1, d_2$, тогда наше уравнение примет вид $d_3 + ... + d_{13} = 12$
    \\
    \\ Так как мы работаем с деревом, то каждая вершина в нем имеет степень не менее 1, следовательно, так как в нашем уравнении осталось 11 слагаемых и все они как минимум 1. То есть минимум их сумма равна 11, чтобы получить 12 нужно прибавть 1 к одному из слагаемых, но в таком случае степень одной из вершин будет равна 2.
    \\
    \\ \textbf{Ответ: } да, следует.
\end{document}