 % Всякое, чтобы работало - все библиотеки
 \documentclass[a4paper, 12pt]{article}
 
 \usepackage[T2A]{fontenc}
 \usepackage[russian, english]{babel}
 \usepackage[utf8]{inputenc}
 \usepackage{subfiles}
 \usepackage{ucs}
 \usepackage{textcomp}
 \usepackage{array}
 \usepackage{indentfirst}
 \usepackage{amsmath}
 \usepackage{amssymb}
 \usepackage{enumerate}
 \usepackage[margin=1.5cm]{geometry}
 \usepackage{authblk}
 \usepackage{tikz}
 \usepackage{icomma}
 \usepackage{gensymb}
 \usepackage{nicematrix, tikz}
  
 % Всякие мат штуки дополнительные
  
 \newcommand{\F}{\mathbb{F}}
 \newcommand{\di}{\frac}
 \renewcommand{\C}{\mathbb{C}}
 \newcommand{\N} {\mathbb{N}}
 \newcommand{\Z} {\mathbb{Z}}
 \newcommand{\R} {\mathbb{R}}
 \newcommand{\Q}{\mathbb{Q}}
 \newcommand{\ord} {\mathop{\rm ord}}
 \newcommand{\Ima}{\mathop{\rm Im}}
 \newcommand{\Rea}{\mathop{\rm Re}}
 \newcommand{\rk}{\mathop{\rm rk}}
 \newcommand{\arccosh}{\mathop{\rm arccosh}}
 \newcommand{\lker}{\mathop{\rm lker}}
 \newcommand{\rker}{\mathop{\rm rker}}
 \newcommand{\tr}{\mathop{\rm tr}}
 \newcommand{\St}{\mathop{\rm St}}
 \newcommand{\Mat}{\mathop{\rm Mat}}
 \newcommand{\grad}{\mathop{\rm grad}}
 \DeclareMathOperator{\spec}{spec}
 \renewcommand{\baselinestretch}{1.5}
 \everymath{\displaystyle}
  
 % Всякое для ускорения
 \renewcommand{\r}{\right}
 \renewcommand{\l}{\left}
 \newcommand{\Lra}{\Leftrightarrow}
 \newcommand{\ra}{\rightarrow}
 \newcommand{\la}{\leftarrow}
 \newcommand{\sm}{\setminus}
 \newcommand{\lm}{\lambda}
 \newcommand{\Sum}[2]{\overset{#2}{\underset{#1}{\sum}}}
 \newcommand{\Lim}[2]{\lim\limits_{#1 \rightarrow #2}}
 \newcommand{\p}[2]{\frac{\partial #1}{\partial #2}}
  
 % Заголовки
 \newcommand{\task}[1] {\noindent \textbf{Задача #1.} \hfill}
 \newcommand{\note}[1] {\noindent \textbf{Примечание #1.} \hfill}
  
 % Пространтсва для задач
 \newenvironment{proof}[1][Доказательство]{%
 \begin{trivlist}
     \item[\hskip \labelsep {\bfseries #1:}]
     \item \hspace{15pt}
     }{
     $ \hfill\blacksquare $
 \end{trivlist}
 \hfill\break
 }
 \newenvironment{solution}[1][Решение]{%
 \begin{trivlist}
     \item[\hskip \labelsep {\bfseries #1:}]
     \item \hspace{15pt}
     }{
 \end{trivlist}
 }
  
 \newenvironment{answer}[1][Ответ]{%
 \begin{trivlist}
     \item[\hskip \labelsep {\bfseries #1:}] \hskip \labelsep
     }{
 \end{trivlist}
 \hfill
 }
 \title{Дз по линейной алгебре 2 Смирнов Тимофей 236 ПМИ}
 \author{Тимофей Смирнов}
 \date{September 2023}
 
 \begin{document}
    {\center \bf \large ДЗ по дискретной математике 9 Смирнов Тимофей 236 ПМИ}
    \\
    \\ \textbf{ДЗ9.1} Рассмотрим бесконечные последовательности из 0, 1 и 2, в которых никакая цифра не встречается два раза подряд. Верно ли, что мощность множества таких последовательностей имеет мощность континуум?
    \\
    \\ \textbf{Решение: } Чтобы доказать, что множество имеет мощность континуум, то необходимо построить биекцию между этим множеством континуумом.
    \\
    \\ Пусть А - множество бесконечных последовательностей из 0, 1 и 2 без повторяющихся символов. 
    \\
    \par 1). Построим инъективную функцию $f: \{0, 1\}^{\mathbb{N}} \to A$. В ней после каждого символа последовательности нулей и единиц мы допишем двойку, если две последовательости 0 и 1 не были равны, то и послучившиеся последовательности не будут равны. Получившиеся \\ последовательности будут принадлежать множеству А. Получаем, что функция $f - $ инъекция.
    \\
    \par 2). Построим инъективную функцию $f: A \to \{0, 1\}^{\mathbb{N}}$. В ней каждую 2 можно заменить на 111, каждый 0 на 10 и каждую единицу на 10. Это будет взаимооднозначным соответствием, в таком случае все строки из А войдут в множество $\{0, 1\}^{\mathbb{N}}$, это будет инъекцией.
    \\
    \\ Мы доказали, что мы можем построить инъекцию из $A$ в $\{0, 1\}^{\mathbb{N}}$ и наоборотм. То есть по теореме Бернштейна между этими множествами существует биекция. То есть А имеет мощность континуума. ЧТД.
    \\
    \\ \textbf{ДЗ9.2} Рассмотрим множество пар различных действительных чисел, то есть 
    \\ \[D = \{(x, y): x \neq y, x,y \in \mathbb{R} \}\]
    \\ Является ли множество D континуальным?
    \\
    \\ \textbf{Решение: } На лекции мы доказывали, что множество действительных чисел равномощно континууму. Так же на лекции мы доказали, что $|\{0, 1\}^{\mathbb{N}} \times \{0, 1\}^{\mathbb{N}}| = |\{0, 1\}^{\mathbb{N}}|$. 
    \\ 
    \\ Рассмотрим множество А - подмножество множества $D$. Пусть $A = \{(x, 1): x \in (\mathbb{R} \setminus 1) \}$. Это множество очевидно равно по мощности множеству действительных чисел (мы просто удалили один элемент из бесконечного множества). То есть А континуально.
    \\ 
    \\ $A \subset D$, следовательно, так как множество А имеет мощность континуум, то мы имеем инъекцтивную функцию $f: \{0, 1\}^{\mathbb{N}} \to D$. Аналогично D является подмножеством $\mathbb{R} \times \mathbb{R}$ (что тоже континуум), то есть мы так же имеем инъекцтивную функцию $g: D \to \{0, 1\}^{\mathbb{N}}$.
    \\
    \\ Мы построили инъекцию в обе стороны, следовательно, по теореме Бернштейна, мы имеем биекцию между множествами $D$ и $\{0, 1\}^{\mathbb{N}}$. То есть D имеет мощность континуум.
    \\
    \\ \textbf{ДЗ9.3} Является ли множество всех тотальных функций $\mathbb{R} \to \mathbb{R}$ континуальным?
    \\
    \\ \textbf{Решение: } на лекции мы доказали, что множество подмножеств некоторого множетсва по мощности больше этого множества.
    \\
    \\ Из этого мы знаем, что $|\mathbb{R}| < |\{0, 1\}^{\mathbb{R}}|$, так как множество подмножеств действительных чисел можно представить в виде $|\{0, 1\}^{\mathbb{R}}|$
    \\
    \\ $|\{0, 1\}^{\mathbb{R}}| < |\mathbb{R}^{\mathbb{R}}|$, так как тут просто на каждой позиции может стоять не 0 или 1, а любое действительное число.
    \\
    \\ Так же мы знаем, что $|\{0, 1\}^{\mathbb{N}}| = |\mathbb{R}|$, следовательно, получаем неравенство: 
    \[|\{0, 1\}^{\mathbb{N}}| = |\mathbb{R}| < |\{0, 1\}^{\mathbb{R}}| < |\mathbb{R}^{\mathbb{R}}|\]
    \\ Из этого следует, что $|\{0, 1\}^{\mathbb{N}}| < |\mathbb{R}^{\mathbb{R}}|$. То есть тотальных функций из $\mathbb{R}$ в $\mathbb{R}$ больше, чем континуум.
    \\
    \\ \textbf{ДЗ9.4} Функция периодическая, если для некоторого числа $T > 0$ (периода) и любого $x$ выполняется
    $f(x+T) = f(x)$. Счётно ли множество множество периодических функций $f: \mathbb{Q} \to \mathbb{Q}$? Период считайте
    рациональным.
    \\
    \\ \textbf{Решение: } Заметим, что для подсчета количества периодических функций с периодом $T$ нам достаточно посчитать количество функций $f : [0, T] \to \mathbb{Q}$, так как их значения будут повторяться с периодом $T$.
    \\
    \\ Из лекции мы знаем, что количество действительных чисел на отрезке $[0, T]$ равно континууму. То есть $|[0, T]| = |\mathbb{Q}|$. То есть для каждого Т нам нужно посчитать количество функций $f: \mathbb{Q} \to \mathbb{Q}$.
    \\
    \\ На лекции мы доказали, что $\mathbb{N}^{\mathbb{N}}$ несчетно, но $|\mathbb{N}| = |\mathbb{Q}| \ \Rightarrow \ \mathbb{Q}^{\mathbb{Q}}$ тоже несчетно.
    \\
    \\ Количество функций $f: \mathbb{Q} \to \mathbb{Q}$ равно $\mathbb{Q}^{\mathbb{Q}}$ (вообще это количество тотальныъ функций, но, если мы добавим в $\mathbb{Q}$ еще один элемент, то ничего не изменится, так что это количество всех функций). Следовательно, уже для какого-то одного Т мы имеем несчетное (континуальное) множество таких функций, а у нас счетное количество различных Т.
    \\
    \\ Получаем, что таких функций у нас несчетное количетсво.
    \end{document}