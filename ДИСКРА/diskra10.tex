 % Всякое, чтобы работало - все библиотеки
 \documentclass[a4paper, 12pt]{article}
 
 \usepackage[T2A]{fontenc}
 \usepackage[russian, english]{babel}
 \usepackage[utf8]{inputenc}
 \usepackage{subfiles}
 \usepackage{ucs}
 \usepackage{textcomp}
 \usepackage{array}
 \usepackage{indentfirst}
 \usepackage{amsmath}
 \usepackage{amssymb}
 \usepackage{enumerate}
 \usepackage[margin=1.5cm]{geometry}
 \usepackage{authblk}
 \usepackage{tikz}
 \usepackage{icomma}
 \usepackage{gensymb}
 \usepackage{nicematrix, tikz}
 \usepackage{tikz}
  
 % Всякие мат штуки дополнительные
  
 \newcommand{\F}{\mathbb{F}}
 \newcommand{\di}{\frac}
 \renewcommand{\C}{\mathbb{C}}
 \newcommand{\N} {\mathbb{N}}
 \newcommand{\Z} {\mathbb{Z}}
 \newcommand{\R} {\mathbb{R}}
 \newcommand{\Q}{\mathbb{Q}}
 \newcommand{\ord} {\mathop{\rm ord}}
 \newcommand{\Ima}{\mathop{\rm Im}}
 \newcommand{\Rea}{\mathop{\rm Re}}
 \newcommand{\rk}{\mathop{\rm rk}}
 \newcommand{\arccosh}{\mathop{\rm arccosh}}
 \newcommand{\lker}{\mathop{\rm lker}}
 \newcommand{\rker}{\mathop{\rm rker}}
 \newcommand{\tr}{\mathop{\rm tr}}
 \newcommand{\St}{\mathop{\rm St}}
 \newcommand{\Mat}{\mathop{\rm Mat}}
 \newcommand{\grad}{\mathop{\rm grad}}
 \DeclareMathOperator{\spec}{spec}
 \renewcommand{\baselinestretch}{1.5}
 \everymath{\displaystyle}
  
 % Всякое для ускорения
 \renewcommand{\r}{\right}
 \renewcommand{\l}{\left}
 \newcommand{\Lra}{\Leftrightarrow}
 \newcommand{\ra}{\rightarrow}
 \newcommand{\la}{\leftarrow}
 \newcommand{\sm}{\setminus}
 \newcommand{\lm}{\lambda}
 \newcommand{\Sum}[2]{\overset{#2}{\underset{#1}{\sum}}}
 \newcommand{\Lim}[2]{\lim\limits_{#1 \rightarrow #2}}
 \newcommand{\p}[2]{\frac{\partial #1}{\partial #2}}
  
 % Заголовки
 \newcommand{\task}[1] {\noindent \textbf{Задача #1.} \hfill}
 \newcommand{\note}[1] {\noindent \textbf{Примечание #1.} \hfill}
  
 % Пространтсва для задач
 \newenvironment{proof}[1][Доказательство]{%
 \begin{trivlist}
     \item[\hskip \labelsep {\bfseries #1:}]
     \item \hspace{15pt}
     }{
     $ \hfill\blacksquare $
 \end{trivlist}
 \hfill\break
 }
 \newenvironment{solution}[1][Решение]{%
 \begin{trivlist}
     \item[\hskip \labelsep {\bfseries #1:}]
     \item \hspace{15pt}
     }{
 \end{trivlist}
 }
  
 \newenvironment{answer}[1][Ответ]{%
 \begin{trivlist}
     \item[\hskip \labelsep {\bfseries #1:}] \hskip \labelsep
     }{
 \end{trivlist}
 \hfill
 }
 \title{Дз по линейной алгебре 2 Смирнов Тимофей 236 ПМИ}
 \author{Тимофей Смирнов}
 \date{September 2023}
 
 \begin{document}
    {\center \bf \large ДЗ по дискретной математике 9 Смирнов Тимофей 236 ПМИ}
    \\
    \\ \textbf{Д10.1} Существует ли граф на 10 вершинах, степени которых равны 1, 1, 1, 1, 1, 1, 3, 5, 5, 5?
    \\
    \\ \textbf{Решение: }
    \\ Да существует, вот пример:
    \\
    $\begin{tikzpicture}[node distance={15mm}, thick, main/.style = {draw, circle}] 
        \node[main] (1) {$1$}; 
        \node[main] (2) [above right of=1] {$2$}; 
        \node[main] (3) [above right of=2] {$3$}; 
        \node[main] (4) [right of=3] {$4$}; 
        \node[main] (5) [below right of=4] {$5$}; 
        \node[main] (6) [below right of=5] {$6$}; 
        \node[main] (7) [below left of=6] {$7$}; 
        \node[main] (8) [below left of=7] {$8$}; 
        \node[main] (9) [left of=8] {$9$}; 
        \node[main] (10) [above left of=9] {$10$}; 
        \draw[-] (1) -- (2); 
        \draw[-] (3) -- (4); 
        \draw(2) to [out=90, in=135,looseness=2] (4);
        \draw(1) to [out=90, in=180,looseness=1] (3); 
        \draw(1) to [out=135, in=90,looseness=1] (4);
        \draw[-] (1) -- (5);
        \draw(1) to [out=270, in=270,looseness=2] (6);
        \draw[-] (2) -- (3);
        \draw[-] (2) -- (7);
        \draw[-] (2) -- (8);
        \draw[-] (3) -- (9);
        \draw[-] (3) -- (10);
      \end{tikzpicture}$
      \\
      \\ Вершина 1 связана с 2, 3, 4, 5, 6 (5 ребер)
      \\ Вершина 2 связана с 1, 3, 4, 7, 8 (5 ребер)
      \\ Вершина 3 связана с 1, 2, 4, 9, 10 (5 ребер)
      \\ Вершины 4 связана с 1, 2, 3 (3 ребра)
      \\ Вершины 7, 8 связаны с 2. (1 ребро)
      \\ Верщины 9, 10 связаны с 3. (1 ребро)
      \\
    \\
    \\ \textbf{Д10.2} Найдите наименьшее количество вершин в графе, сумма степеней вершин в котором равна 26.
    \\
    \\ \textbf{Решение: } Сумма степеней вершин в графе равна удвоенной сумме ребер этого графа, следовательно данный граф имеет 13 ребер.
    \\
    \\ Если бы у нас в графе было 5 вершин, то даже если бы он был полный, то там не могло бы быть больше 10 ребер (логично, что если вершин меньше 5, то максиммально ребер в них было бы еще меньше, чем 10). Но нам необходимо ровно 13 ребер, следовательно в данном графе как минимум 6 вершин.
    \\
    \\ Приведем пример графа с 6ю вершинами и 13ю ребрами:
    \\
    $\begin{tikzpicture}[node distance={15mm}, thick, main/.style = {draw, circle}] 
        \node[main] (1) {$1$}; 
        \node[main] (2) [above right of=1] {$2$}; 
        \node[main] (3) [above right of=2] {$3$}; 
        \node[main] (4) [right of=3] {$4$}; 
        \node[main] (5) [below right of=4] {$5$}; 
        \node[main] (6) [below right of=5] {$6$}; 
        \draw[-] (1) -- (2); 
        \draw(1) to [out=90, in=180,looseness=1] (3); 
        \draw(6) to [out=90, in=0,looseness=1] (4); 
        \draw(1) to [out=135, in=90,looseness=1] (4);
        \draw(6) to [out=45, in=90,looseness=2] (3);
        \draw[-] (1) -- (5);
        \draw[-] (3) -- (4);
        \draw[-] (5) -- (4);
        \draw[-] (5) -- (6);
        \draw(6) to [out=270, in=270,looseness=1] (1);
        \draw[-] (2) -- (3);
        \draw[-] (2) -- (5);
        \draw[-] (2) -- (6);
      \end{tikzpicture}$
    \\
    \\ Мы построили граф с 13ю ребрами и 6ю вершинами.
    \\
    \\ \textbf{Д10.3} Вершины графа G — слова длины 2 в алфавите $\{0, 1, 2, 3, 4, 5, 6, 7, 8, 9\}$, то есть последовательно-
    сти десятичных цифр длины 2. Две вершины (два слова длины 2) соединены ребром в G, если в каждой из позиций цифры различаются ровно на 1. Найдите количество компонент связности графа G.
    \\
    \\ \textbf{Решение: } Заметим, что на каждом переходе между двемя вершинами мы либо увеличиваем разность между цифрыми в вершинах на 2 (к большей цифре прибавляем 1, а из меньшей вычитаем), либо не изменяем разность (увеличиваем или уменьшаем на 1 обе цифры), либо уменьшаем разноть на 2 (уменьшаем на 1 большую цифру и увеличиваем на 1 меньшую цифру).
    \\
    \\ Из этого следует, что из вершин с нечетной разностью цифр мы сможем прийти только в вершины с нечетной разностью. При этом из вершины с нечетной разностью можно прийти в любую вершину с нечетной разностью, просто придя сначала, например, в вершину (0, 1), а потом придя из нее в любую другую. В вершину (0, 1) мы можем прийти всегда: сначала нужно уменьшить разность между цифрами в вершине до 1 (уменьшаем на 1 большее и увеличиваем на 1 меньшее на каждом шаге), потом сделать так, чтобы первая цифры была меньше 2й на 1 (если этого не было сделано ранее), а затем просто вычитать из обеих цицфр 1, пока мы не окажемся в вершине (0, 1). Обратная последовательность действий приведет нас из вершины (0, 1) в любую вершину с нечетной разностью цифр.
    \\
    \\ Из вершины с четной разностью мы можем прийти сначала в (0, 0): просто уменьшим разность между цифрами вершины до 0 (по алгоритму из предыдущего абзаца), а затем на каждом шаге будем уменьшать обе цифры на 1. Обратной последовательностью действий мы сможем прийти из (0, 0) в любую вершину с четной разностью цифр.
    \\
    \\ Алгоритмы выше показали, что через вершину (0, 1) мы сможем добраться от любой вершины с нечетной разностью до любой другой вершины с нечетной разностью. Аналогично при помощи вершины (0, 0) мы доберемся до любой вершины с четной разностью.
    \\
    \\ При этем из вершин с четной разностью мы никак не сможем попасть в вершины с нечетной разностью и наоборот
    \\
    \\ Следовательно наш граф поделится на 2 компоненты связности.
    \\
    \\ \textbf{Ответ: } 2.
    \\
    \\ \textbf{Д10.4} Пусть $A$ — непустое множество, $E1$ и $E2$ — такие отношения эквивалентности на $A$, что $E_1 \cup E_2$ также является отношением эквивалентности, 
    $C_1$ — класс эквивалентности отношения $E_1, C_2$ — класс эквивалентности отношения $E_2$. Докажите, что $C_1 \cap C_2 = \varnothing$, или $C_1 \subseteq C_2$, или $C_2 \subseteq C_1$.
    \\
    \\ \textbf{Решение: }
    \\ $C_1 = \{x: \forall y \in A \hookrightarrow xE_1y\}, \ C_2 = \{x: \forall y \in A \hookrightarrow xE_2y\}$
    \\
    \\ Пусть $E_1 \cup E_2$ является отношением эквивалентности. Рассмотрим варианты, того, как могут вести себя $C_1$ и  $C_2$:
    \\
    \par 1). Пусть $C_1 \cup C_2 = \varnothing$. Тогда, так как любая пара элементов из $C_1$ лежит в отношении эквивалентности $E_1$ на множестве А и любая пара из $C_2$ лежит в отношении эквиввалентности $E_2$ на множестве А, то есть любая пара элементов из $C_1$ или пара элеентов из $C_2$ лежат в отношении эквивалентности $E_1 \cup E_2$ При этом они не пересекаются, то есть они являются классами эквивалентности $E_1 \cup E_2$.
    \\
    \par 2). Пусть $C_1 \subseteq C_2$ (или $C_2 \subseteq C_1$, доказательство точно такое же, так что его я опущу). Для любой пары из $C_1 \cup C_2$ будут выполняться свойства отношения жквивалентности, так как все элементы $C_1$ лежат в множестве $C_2$, а для этого множества все эти свойства выполняются.
    \\
    \par 3). Пусть $C_1 \cup C_2 \neq \varnothing$. Тогда $\exists z \in C_1 \cap C_2$. Тогда, так как $E_1 \cup E_2$ является отношением эквивалентности, то для него выполняется свойство транзитивности. Пусть $x \in C_1$ и $y \in C_2$, тогда так как $xE_1z$ и $zE_2y$ верны, то $xRy$ верно по некоторому отношению эквивалентности. Но $xE_1z$ и $zE_2y$ у нас верны по разным отношениям эквивалентности и у нас не существует свойства, по которому будут эквиваленты $x$ и $y$. Получаем противоречие: в этом случае $E_1 \cup E_2$ не будет отношением эквивалентности.
\end{document}