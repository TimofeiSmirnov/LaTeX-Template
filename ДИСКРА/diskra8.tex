 % Всякое, чтобы работало - все библиотеки
 \documentclass[a4paper, 12pt]{article}
 
 \usepackage[T2A]{fontenc}
 \usepackage[russian, english]{babel}
 \usepackage[utf8]{inputenc}
 \usepackage{subfiles}
 \usepackage{ucs}
 \usepackage{textcomp}
 \usepackage{array}
 \usepackage{indentfirst}
 \usepackage{amsmath}
 \usepackage{amssymb}
 \usepackage{enumerate}
 \usepackage[margin=1.5cm]{geometry}
 \usepackage{authblk}
 \usepackage{tikz}
 \usepackage{icomma}
 \usepackage{gensymb}
 \usepackage{nicematrix, tikz}
  
 % Всякие мат штуки дополнительные
  
 \newcommand{\F}{\mathbb{F}}
 \newcommand{\di}{\frac}
 \renewcommand{\C}{\mathbb{C}}
 \newcommand{\N} {\mathbb{N}}
 \newcommand{\Z} {\mathbb{Z}}
 \newcommand{\R} {\mathbb{R}}
 \newcommand{\Q}{\mathbb{Q}}
 \newcommand{\ord} {\mathop{\rm ord}}
 \newcommand{\Ima}{\mathop{\rm Im}}
 \newcommand{\Rea}{\mathop{\rm Re}}
 \newcommand{\rk}{\mathop{\rm rk}}
 \newcommand{\arccosh}{\mathop{\rm arccosh}}
 \newcommand{\lker}{\mathop{\rm lker}}
 \newcommand{\rker}{\mathop{\rm rker}}
 \newcommand{\tr}{\mathop{\rm tr}}
 \newcommand{\St}{\mathop{\rm St}}
 \newcommand{\Mat}{\mathop{\rm Mat}}
 \newcommand{\grad}{\mathop{\rm grad}}
 \DeclareMathOperator{\spec}{spec}
 \renewcommand{\baselinestretch}{1.5}
 \everymath{\displaystyle}
  
 % Всякое для ускорения
 \renewcommand{\r}{\right}
 \renewcommand{\l}{\left}
 \newcommand{\Lra}{\Leftrightarrow}
 \newcommand{\ra}{\rightarrow}
 \newcommand{\la}{\leftarrow}
 \newcommand{\sm}{\setminus}
 \newcommand{\lm}{\lambda}
 \newcommand{\Sum}[2]{\overset{#2}{\underset{#1}{\sum}}}
 \newcommand{\Lim}[2]{\lim\limits_{#1 \rightarrow #2}}
 \newcommand{\p}[2]{\frac{\partial #1}{\partial #2}}
  
 % Заголовки
 \newcommand{\task}[1] {\noindent \textbf{Задача #1.} \hfill}
 \newcommand{\note}[1] {\noindent \textbf{Примечание #1.} \hfill}
  
 % Пространтсва для задач
 \newenvironment{proof}[1][Доказательство]{%
 \begin{trivlist}
     \item[\hskip \labelsep {\bfseries #1:}]
     \item \hspace{15pt}
     }{
     $ \hfill\blacksquare $
 \end{trivlist}
 \hfill\break
 }
 \newenvironment{solution}[1][Решение]{%
 \begin{trivlist}
     \item[\hskip \labelsep {\bfseries #1:}]
     \item \hspace{15pt}
     }{
 \end{trivlist}
 }
  
 \newenvironment{answer}[1][Ответ]{%
 \begin{trivlist}
     \item[\hskip \labelsep {\bfseries #1:}] \hskip \labelsep
     }{
 \end{trivlist}
 \hfill
 }
 \title{Дз по линейной алгебре 2 Смирнов Тимофей 236 ПМИ}
 \author{Тимофей Смирнов}
 \date{September 2023}
 
 \begin{document}
    {\center \bf \large ДЗ по дискретной математике 8 Смирнов Тимофей 236 ПМИ}
    \\ \textbf{Д8.1}
    \\
    \par а) Верно ли, что если $|A \setminus B| = |B \setminus A|$, то $|A| = |B|$?
    \\
    \\ \textbf{Решение: } $|A \setminus B| = |A| - |A \cap B|, \ \ |B \setminus A| = |B| - |B \cap A|;$
    \\ Из условия получаем, что $|A| - |A \cap B| = |B| - |B \cap A| \ \Leftrightarrow \ |A| - |B| = |A \cap B| - |B \cap A|;$
    \\ $|A \cap B| - |B \cap A| = 0 \ \Rightarrow \ |A| - |B| = 0$
    \\
    \par \ \ \ \ \textbf{Ответ: } \textbf{Верно}, что если $|A \setminus B| = |B \setminus A|$, то $|A| = |B|$.
    \\
    \par б). Верно ли, что если $|A| = |C|, \ |B| = |D|, \ B \in A, \ D \in C, \ $ то $ |A \setminus B| = |C \setminus D|$?
    \\
    \\ \textbf{Решение: } $|A \setminus B| = |A| - |A \cap B|, \ \ |C \setminus D| = |C| - |C \cap D|$
    \\ Так как $B \in A$ и $ D \in C \ \Rightarrow \ A \cap B = B$ и $ C \cap D = D$, получаем $|A \setminus B| = |A| - |B|, \ \ |C \setminus D| = |C| - |D|$
    \\
    \\ $|A| = |C|$ и $\ |B| = |D| \ \Rightarrow \ |A| - |B| = |C| - |D|$
    \\
    \par \ \ \ \ \textbf{Ответ: } \textbf{Верно}, что если $|A| = |C|, \ |B| = |D|, \ B \in A, \ D \in C, \ $ то $ |A \setminus B| = |C \setminus D|$
    \\
    \\ \textbf{Д8.2} Докажите, что множество конечных подмножеств рациональных чисел счётно.
    \\
    \\ \textbf{Решение: } любое конечное подмножество рациональных чисел длины $k$ является элементом декартова произведения $\mathbb{Q} \times \mathbb{Q} \times ... \times \mathbb{Q}$ с $k$ множителями. Следовательно, множество подмножеств длины $k$ рациональных чисел является подмножеством множества $\mathbb{Q}^k$ (именно подмножеством, так как в декартовом произведении на разных позициях в одной последовательности могут стоять одинаковые элементы, а в нужном нам множестве длины k нам нужно k различных элементов).
    \\
    \\ Так как множество рациональных чисел счетно, то и множество $\mathbb{Q}^k$ счетно, а значит и любое подмножество $\mathbb{Q}^k$ счетно.
    \\
    \\ Мы можем построить биекцию между натуральными числами и такими множествами подмножеств рациональных чисел: каждому числу $k \in \mathbb{N}$ будет соответствовать множество подмножеств длины $k$ рациональных чисел .
    \\    
    \\ Так как $\mathbb{N}$ счетно, то объединяя все такие подмножеств рациональных чисел мы получаем объединение счестного числа счетных множеств, что является счетным множеством. ЧТД.
    \\
    \\ \textbf{Д8.3} Функция периодическая, если для некоторого числа $T$ и любого $x$ выполняется $f(x + T) = f(x)$.
    Докажите, что множество периодических функций $f: \mathbb{Z} \to \mathbb{Z}$ счётно. (Число Т целом положительное*)
    \\
    \\ \textbf{Решение: } Так как $T \in \mathbb{N}$, то мы имеем счетное количество возможных периодов Т.
    \\ Заметим, что для подсчета количества периодических функций с периодом Т нам достаточно посчитать количество тотальных функций $f: [T] \to \mathbb{Z}$, так как их значения будут повторяться с периодом Т.
    \\
    \\ Количество тотальных функций $f: [T] \to \mathbb{Z}$ равно $|\mathbb{Z}^T|$, так как функции не должны быть исключительно инъективными.
    \\
    \\ Так как множество целых чисел счетно, то $\forall \ T \in \mathbb{N} \ \hookrightarrow \ \mathbb{Z}^T$ счетно. Но так как возможных Т у нас тоже счетное количество, то мы получаем объединение счетного числа счетных множеств, что является счетным множеством. Следовательно таких периодических функций счетное количество. ЧТД.
    \\
    \\ \textbf{Д8.4} Тотальную функцию из $\mathbb{N}$ в $\mathbb{N}$ назовём представительной, если она строго возрастающая и её
    значения — все натуральные числа за исключением конечного множества. Докажите, что множество
    представительных функций счётно.
    \\
    \\  \textbf{Решение: } Если наша представительная функция строго возрастает, то она инъективна. Так как, если $\exists x_1 \neq x_2 : f(x_1) = f(x_2),$ то функция была бы как минимум неубывающей. 
    \\
    \\ Так же по условию задачи сказано, что она определена на всей области значений (натуральные числа без конечного множества), следовательно, по условию она сюрьективна. Из этого следует, что она биективна.
    \\
    \\ Для любого конечного подмножества натуральных чисел такая функция единственна. Так как, если она строго возрастает и биективна, то она задается последовательностью $a_0 a_1 a_2 ...$, где $a_0 < a_1 < a_2 < ...$, и в этой последовательность содержатся все натуральные числа без какого-то конечного можества, то есть они просто расположены по возрастанию, если изменить последовательность, то она перестанет быть возрастающей.
    \\
    \\ В итоге мы получаем, что каждому конечному подмножеству натуральных чисел соответствует ровно одна представительная функция. Во второй задаче мы доказали, что количество конечных подмножеств вещественных чисел счетно, следовательно счетно и количество конечных подмножеств натуральных чисел. Получается, что так как для каждого конечного подмножества натуральных чисел такая функция единственна, то всего таких функций счетное количество, так как таких подмножеств счетное количество. ЧТД
    \\
    \\ \textbf{Д8.5} Множество A состоит из бесконечных последовательностей десятичных цифр (то есть, элементов
    множества $\{0, 1, . . . , 9\}$), в которых цифра 5 встречается на втором месте, а больше эта цифра нигде
    не встречается. Является ли это множество счётным?
    \\
    \\ \textbf{Решение: } рассмотрим множество таких последовательностей и расположим их в таблице в таком виде:
    \[
        \begin{matrix}
            A_1 & a_{11} & 5 & a_{13} & a_{14} & \dots \\
            A_2 & a_{21} & 5 & a_{23} & a_{24} & \dots \\
            A_3 & a_{31} & 5 & a_{33} & a_{34} & \dots \\
            A_4 & a_{41} & 5 & a_{43} & a_{44} & \dots \\
        \end{matrix}     
    \]
    \\
    \\ Здесь $A_i$ обозначает $i-$ю последовательность, а бесконечный ряд чисел $a_{i1}, a_{i2} ...$ это ее члены
    \\
    \\ Рассмотрим диагональную последовательность $a_{11} a_{22} a_{33} ... $ и преобразуем ее двумя способами:
    \par \ \ \ \ 1). Все элементы кроме второго (пятерки) заменим по правилу $a_{ii} = 1 - (a_{ii} \mod 2)$. В таком слечае каждое четное число превратится в 1, а нечетное в 0. То есть все цифры изменятся.
    \par \ \ \ \ 2). Все элементы кроме второго (пятерки) заменим по правилу $a_{ii} = 9 - (a_{ii} \mod 4)$. В таком случае тоже изменятся все цифры.
    \\
    \\ При поервом преобозовании мы получим некоторую последовательность $b = (b_1 b_2 b_3 \dots)$, а при втором преобовазовании мы получим отличную от нее последовательность $c = (c_1 c_2 c_3 \dots)$
    \\
    \\ Обе этим последовательности будут заведомо отличаться от любой другой последовательности в таблице кроме последовательности $A_2$, эта последовательноть может совпасть с одной из полученных последовательностей $b$ или $c$, но так как их 2 и они не совпадают, то хотя бы одна из них не совпадет ни с одной из последовательностей таблицы. Получаем, что $b \notin \mathbb{N}$ или $c \notin \mathbb{N}$, из чего следует, что ножество таких последовательностей нечетно.
    \end{document}