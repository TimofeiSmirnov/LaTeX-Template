 % Всякое, чтобы работало - все библиотеки
 \documentclass[a4paper, 12pt]{article}
 
 \usepackage[T2A]{fontenc}
 \usepackage[russian, english]{babel}
 \usepackage[utf8]{inputenc}
 \usepackage{subfiles}
 \usepackage{ucs}
 \usepackage{textcomp}
 \usepackage{array}
 \usepackage{indentfirst}
 \usepackage{amsmath}
 \usepackage{amssymb}
 \usepackage{enumerate}
 \usepackage[margin=1.5cm]{geometry}
 \usepackage{authblk}
 \usepackage{tikz}
 \usepackage{icomma}
 \usepackage{gensymb}
 \usepackage{nicematrix, tikz}
 \usepackage{tikz}
  
 % Всякие мат штуки дополнительные
  
 \newcommand{\F}{\mathbb{F}}
 \newcommand{\di}{\frac}
 \renewcommand{\C}{\mathbb{C}}
 \newcommand{\N} {\mathbb{N}}
 \newcommand{\Z} {\mathbb{Z}}
 \newcommand{\R} {\mathbb{R}}
 \newcommand{\Q}{\mathbb{Q}}
 \newcommand{\ord} {\mathop{\rm ord}}
 \newcommand{\Ima}{\mathop{\rm Im}}
 \newcommand{\Rea}{\mathop{\rm Re}}
 \newcommand{\rk}{\mathop{\rm rk}}
 \newcommand{\arccosh}{\mathop{\rm arccosh}}
 \newcommand{\lker}{\mathop{\rm lker}}
 \newcommand{\rker}{\mathop{\rm rker}}
 \newcommand{\tr}{\mathop{\rm tr}}
 \newcommand{\St}{\mathop{\rm St}}
 \newcommand{\Mat}{\mathop{\rm Mat}}
 \newcommand{\grad}{\mathop{\rm grad}}
 \DeclareMathOperator{\spec}{spec}
 \renewcommand{\baselinestretch}{1.5}
 \everymath{\displaystyle}
  
 % Всякое для ускорения
 \renewcommand{\r}{\right}
 \renewcommand{\l}{\left}
 \newcommand{\Lra}{\Leftrightarrow}
 \newcommand{\ra}{\rightarrow}
 \newcommand{\la}{\leftarrow}
 \newcommand{\sm}{\setminus}
 \newcommand{\lm}{\lambda}
 \newcommand{\Sum}[2]{\overset{#2}{\underset{#1}{\sum}}}
 \newcommand{\Lim}[2]{\lim\limits_{#1 \rightarrow #2}}
 \newcommand{\p}[2]{\frac{\partial #1}{\partial #2}}
  
 % Заголовки
 \newcommand{\task}[1] {\noindent \textbf{Задача #1.} \hfill}
 \newcommand{\note}[1] {\noindent \textbf{Примечание #1.} \hfill}
  
 % Пространтсва для задач
 \newenvironment{proof}[1][Доказательство]{%
 \begin{trivlist}
     \item[\hskip \labelsep {\bfseries #1:}]
     \item \hspace{15pt}
     }{
     $ \hfill\blacksquare $
 \end{trivlist}
 \hfill\break
 }
 \newenvironment{solution}[1][Решение]{%
 \begin{trivlist}
     \item[\hskip \labelsep {\bfseries #1:}]
     \item \hspace{15pt}
     }{
 \end{trivlist}
 }
  
 \newenvironment{answer}[1][Ответ]{%
 \begin{trivlist}
     \item[\hskip \labelsep {\bfseries #1:}] \hskip \labelsep
     }{
 \end{trivlist}
 \hfill
 }
 \title{Дз по линейной алгебре 2 Смирнов Тимофей 236 ПМИ}
 \author{Тимофей Смирнов}
 \date{September 2023}
 
 \begin{document}
    {\center \bf \large ДЗ по дискретной математике 13 Смирнов Тимофей 236 ПМИ}
    \\
    \\ \textbf{Д13.1} Простой неориентированный граф G можно по крайней мере тремя различными способами
    правильно раскрасить в 2 цвета. Докажите, что G несвязный.
    \\
    \\ \textbf{Решение: } Рассмотрим связный граф, который можно раскрасить в 2 цвета, каждую его вершину можно раскрасить лишь в 2 различных цвета. Причем по построению раскраски связного графа в 2 цвета в теореме 13.6 раскраска всего графа целиком зависит от раскраски первой выбранной вершины. Так как первую вершину можно раскрасить всего в 2 цвета, следовательно и связный граф можно раскрасить всего двумя способами. То есть связный граф нельзя раскрасить в 2 цвета более чем двумя способами.
    \\
    \\ \textbf{А вот несвязный граф можно:} Пусть в нашем несвязном графе 2 компоненты связности (с большим количеством компонент все тоже будет выполняться), тогда, если каждую из этих компонент можно раскрасить в 2 цвета, то каждую из них можно раскрасить двумя способами (просто инвертировав цвета). При комбинации этих способов действует правило умножения (так как способы раскраски компонент никак друг от друга не зависят). То есть всего наш несвязный граф можно раскрасить $4 = 2 \cdot 2$ способами.
    \\
    \\ \textbf{Ответ: } получается, что связный граф нельзя раскрасить в 2 цвета более чем двумя способами, а несвязный можно по крайней мере тремя (при количестве компонент связности больше двух будет больше комбинаций).
    \\
    \\
    \\ \textbf{Д13.2} Назовём не 2-раскрашиваемый простой неориентированный граф минимальным, если после
    удаления любого ребра он становится 2-раскрашиваемым. Докажите, что в минимальном не 2-раскрашиваемом графе на 1000 вершинах есть хотя бы одна изолированная вершина (то есть вершина
    степени 0).
    \\
    \\ \textbf{Решение: } В определенеии минимального графа удаляется ЛЮБОЕ ребро, следовательно, минимальный не 2-раскрашиваемый граф может состоять только из одного цикла, в который входят все вешины, так как:
    \par 1. В не 2-раскрашиваемом графе есть цикл нечетной длины.
    \par 2. Если он состоить из нескольких циклов длины больше двух, то хотя бы один из этих циклов имеет нечетную длину, а так как циклов несколько, то найдется ребро, при удалении которого в графе останется нечетный цикл. 
    \par 3. Если в графе помимо циклов есть вершины, которые не входят в циклы больше двух, то между этими вершинами (или каким-либо из циклов и этими вершинами) можно найти ребро, при удалении которого в графе все еще останется нечетный цикл.
    \par 4. Так же в графе может быть не более одной компоненты связности, в которую входит более одной вершины, так как если таких компонент будет хотя бы 2, то можно будет не удалять ребро из компоненты с нечетным циклом, а удалить его из другой компоненты.
    \\
    \\ Итак, по условию задачи в нашем графе 1000 вершин. Так как он минимальный не 2-раскрашиваемый, то он состоит из одного большого цикла нечетной длины (следствие из пунктов) и не7которого (возможно нулевого) количества изолированных вершин. Но так как в нашем графе четное число вершин, следовательно все вершины не могут образовывать один нечетный цикл. Получается, что хотя бы одна вершина будет изолированной (по пункту 4).
    \\
    \\ \textbf{Д13.3} Количество вершин в графе G равно 2301. Известно, что в G нет клики размера 4. Докажите,
    что в G есть независимое множество размера 23.
    \\
    \\ \textbf{Решение: } Получим верхнюю оценку числа $R(4, 23)$. По Теореме о верхней оценке чисчел Рамсея получаем $R(4, 23) \leq \binom{23 + 4 - 2}{4 - 1} = \binom{25}{3} = 2300$. 
    \\
    \\ В нашем графе 2301 вершина, следовательно, так как в нем нет клики размера 4, то в нем найдется независимое множество размера 23.
    \\
    \\ \textbf{Д13.4} В графе D существуют два таких остовных дерева $T_1$ и $T_2$, что объединение рёбер этих дере-
    вьев совпадает с множеством рёбер графа D. Докажите, что вершины графа D возможно правильно
    раскрасить в 4 цвета.
    \\
    \\ \textbf{Решение: } Любое дерево является 2-раскрашиваемым. Раскрасим первое остовное дерево в цвета 0 и 1. Раскрасим второе дерево в цвета 0 и 2. 
    \\
    \\ Объединим эти деревья, при этом просуммируем цвета, находящиеся в вершинах деревьев. То есть, если i-я вершина в первом дереве имела цвет 0 и та же вершина во втором дереве имела цвет 2, то цвет итоговой вершины после объединения будет $0 + 2 = 2$. В итоге мы получим исходный граф, раскрашенный в какие-то цвета.
    \\
    \\ Заметим, что в таком дереве вершины, раскрашенные в одинаковые цвета, не будут соединены ребром, так как одинаковые номера цветов в новом дереве мы можем получить только суммированием одинаковых номеров цветов в исходных остовных деревьях. Но исходные деревья 2-раскрашиваемы, следовательно две смежные вершины в них не имеют одинаковых цветов. Следовательно и в получившемся графе не будет двух одинаковых вершин.
    \\
    \\ Всего суммированием цветов (0 и 1 в первом остовном дереве и 0 и 2 во втором) мы можем получить цвета 0, 1, 2, 3. То есть 4 различных цвета. Следовательно в получившемся дереве будет 4 или меньше цветов, при этом две соседние вершины будут раскрашены в разные цвета. Следовательно такое дерево будет иметь правильную раскраску из 4х цветов.
\end{document}