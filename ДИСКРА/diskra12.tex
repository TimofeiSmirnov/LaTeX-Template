 % Всякое, чтобы работало - все библиотеки
 \documentclass[a4paper, 12pt]{article}
 
 \usepackage[T2A]{fontenc}
 \usepackage[russian, english]{babel}
 \usepackage[utf8]{inputenc}
 \usepackage{subfiles}
 \usepackage{ucs}
 \usepackage{textcomp}
 \usepackage{array}
 \usepackage{indentfirst}
 \usepackage{amsmath}
 \usepackage{amssymb}
 \usepackage{enumerate}
 \usepackage[margin=1.5cm]{geometry}
 \usepackage{authblk}
 \usepackage{tikz}
 \usepackage{icomma}
 \usepackage{gensymb}
 \usepackage{nicematrix, tikz}
 \usepackage{tikz}
  
 % Всякие мат штуки дополнительные
  
 \newcommand{\F}{\mathbb{F}}
 \newcommand{\di}{\frac}
 \renewcommand{\C}{\mathbb{C}}
 \newcommand{\N} {\mathbb{N}}
 \newcommand{\Z} {\mathbb{Z}}
 \newcommand{\R} {\mathbb{R}}
 \newcommand{\Q}{\mathbb{Q}}
 \newcommand{\ord} {\mathop{\rm ord}}
 \newcommand{\Ima}{\mathop{\rm Im}}
 \newcommand{\Rea}{\mathop{\rm Re}}
 \newcommand{\rk}{\mathop{\rm rk}}
 \newcommand{\arccosh}{\mathop{\rm arccosh}}
 \newcommand{\lker}{\mathop{\rm lker}}
 \newcommand{\rker}{\mathop{\rm rker}}
 \newcommand{\tr}{\mathop{\rm tr}}
 \newcommand{\St}{\mathop{\rm St}}
 \newcommand{\Mat}{\mathop{\rm Mat}}
 \newcommand{\grad}{\mathop{\rm grad}}
 \DeclareMathOperator{\spec}{spec}
 \renewcommand{\baselinestretch}{1.5}
 \everymath{\displaystyle}
  
 % Всякое для ускорения
 \renewcommand{\r}{\right}
 \renewcommand{\l}{\left}
 \newcommand{\Lra}{\Leftrightarrow}
 \newcommand{\ra}{\rightarrow}
 \newcommand{\la}{\leftarrow}
 \newcommand{\sm}{\setminus}
 \newcommand{\lm}{\lambda}
 \newcommand{\Sum}[2]{\overset{#2}{\underset{#1}{\sum}}}
 \newcommand{\Lim}[2]{\lim\limits_{#1 \rightarrow #2}}
 \newcommand{\p}[2]{\frac{\partial #1}{\partial #2}}
  
 % Заголовки
 \newcommand{\task}[1] {\noindent \textbf{Задача #1.} \hfill}
 \newcommand{\note}[1] {\noindent \textbf{Примечание #1.} \hfill}
  
 % Пространтсва для задач
 \newenvironment{proof}[1][Доказательство]{%
 \begin{trivlist}
     \item[\hskip \labelsep {\bfseries #1:}]
     \item \hspace{15pt}
     }{
     $ \hfill\blacksquare $
 \end{trivlist}
 \hfill\break
 }
 \newenvironment{solution}[1][Решение]{%
 \begin{trivlist}
     \item[\hskip \labelsep {\bfseries #1:}]
     \item \hspace{15pt}
     }{
 \end{trivlist}
 }
  
 \newenvironment{answer}[1][Ответ]{%
 \begin{trivlist}
     \item[\hskip \labelsep {\bfseries #1:}] \hskip \labelsep
     }{
 \end{trivlist}
 \hfill
 }
 \title{Дз по линейной алгебре 2 Смирнов Тимофей 236 ПМИ}
 \author{Тимофей Смирнов}
 \date{September 2023}
 
 \begin{document}
    {\center \bf \large ДЗ по дискретной математике 12 Смирнов Тимофей 236 ПМИ}
    \\
    \\ \textbf{Д12.1} Докажите, что можно так занумеровать вершины связного неориентированного графа на n
    вершинах числами от $1$ до $n$, что для каждого $1 \leq k \leq n$ связен подграф, индуцированный множеством
    вершин с номерами от $1$ до $k$.
    \\
    \\ \textbf{Решение: } Занумеруем вершины. На первом шаге возьмем любую вершину графа (мы еще ни одной не зинумеровали) и присвоим ей номер 1. Так как граф связный, то у вершины как минимум один сосед, пусть таких соседей $h$, тогда пройдемся по ним и присвоим им номера от 2 до $h + 1$. 
    \\
    \\ Далее для каждого из таких соседей запустим алгоритм: будем заходить в вершину и искать незанумерованных соседей в ней, если таковые имеются, то пусть их будет $f$, а уже занумерованных вершин всего $l$, тогда занумеруем этих соседей номерами от $l + 1$ до $l + f$. Если у вершины, в которую мы заходим все соседи занумерованы - то выходим из нее.
    \\
    \\ Обойдя всех соседей первой вершины обходим тем же алгоритмом всех ее соседей, где эти соседи теперь будут "стартовыми" вершинами.
    \\
    \\ Таким образом мы сначала обойдем всех соседей вершины с номером 1 и занумеруем их. Затем всех ее соседей и всех соседей их соседей. После такого алгоритма каждая вершина кроме первой будет иметь соседа с меньшим номером. Следовательно из нее можно будет добраться до вершины с номером 1. А уже из первой вершины можно будет добраться в любую другую.
    \\
    \\ Таким образом индуцированный граф с множеством вершин от 1 до $k$ будет так же связным, так как по построению алгоритма из каждой вершины можно будет попасть в первую, проходя только через вершины меньшего номера.
    \\
    \\ \\ \\ 
    \\ \\ \\ \\ \\ \\ \textbf{12.2} Найдите такой граф на 8 вершинах, что степень каждой вершина равна 3 и в этом графе нет
    независимого множества размера 4. (Напомним, что, как и во всех остальных задачах, ответ должен
    быть обоснован. Нужно доказать, что ваш пример удовлетворяет требуемым свойствам.)
    \\
    \\ \textbf{Решение: }
    \\ Приведем пример такого графа:
    \\ \\ \\
    \[\begin{tikzpicture}[node distance={15mm}, thick, main/.style = {draw, circle}] 
        \node[main] (1) {$1$}; 
        \node[main] (2) [above of=1] {$2$}; 
        \node[main] (3) [above right of=2] {$3$}; 
        \node[main] (4) [right of=3] {$4$}; 
        \node[main] (5) [below right of=4] {$5$}; 
        \node[main] (6) [below of=5] {$6$}; 
        \node[main] (7) [below left of=6] {$7$}; 
        \node[main] (8) [left of=7] {$8$};
        \draw[-] (1) -- (2); 
        \draw[-] (2) -- (3);
        \draw[-] (3) -- (4);
        \draw[-] (4) -- (5);
        \draw[-] (5) -- (6);
        \draw[-] (6) -- (7);
        \draw[-] (7) -- (8);
        \draw[-] (1) -- (8);
        \draw[-] (7) -- (3);
        \draw[-] (4) -- (8);
        \draw[-] (2) -- (5);
        \draw[-] (1) -- (6);
      \end{tikzpicture}\]
    \\
    \\ В этом графе 8 вершин и степень каждой вершины 3. Заметим, что две соседние вершины не могут лежать в независимом множестве, так как они соединены ребром.
    \\
    \\ Из вышесказанного получаем, что в одном независимом множестве из 4х вершин могут лежать либо вершины с номерами 1, 3, 5, 7 или вершины с номерами 2, 4, 6, 8. Но, по построению графа, между вершинами 3 и 7, а так же 4 и 8 существует ребро, следовательно они тоже не могут лежать в одном независимом множестве. Получается, что множества вершин с номерами 1, 3, 5, 7 не является независимым множеством и множество вершин с номерами 2, 4, 6, 8 тоже не является независимым множеством. Следовательно независимых множеств на 4х вершинах в графе нет.
    \\
    \\ \\ \\ \\ \\ \\ \\ \textbf{Д12.3} Известно, что в простом неориентированном графе нечётное количество независимых множеств. Следует ли из этого, что граф связный? (Независимое множество — это подмножество вершин, в кото-
    ром каждая пара вершин не соединена ребром. Пустое множество и 1-элементные множества являются
    независимыми.)
    \\
    \\ \textbf{Решение: } 
    \\
    \\ Приведем пример несвязного графа с нечетным количеством независимых множеств:
    \\
    \[
        \begin{tikzpicture}[node distance={15mm}, thick, main/.style = {draw, circle}] 
            \node[main] (1) {$1$}; 
            \node[main] (2) [above of=1] {$2$}; 
            \node[main] (3) [right of=2] {$3$}; 
            \node[main] (4) [below of=3] {$4$}; 
            \draw[-] (1) -- (2); 
            \draw[-] (3) -- (4);
          \end{tikzpicture}
    \]
    \\ Перечислим независимые множества этого графа: $\{\varnothing\}, \{1\}, \{2\}, \{3\}, \{4\}, \{1, 3\}, \{1, 4\}, \{2, 3\}, \{2, 4\}$.
    \\
    \\ Всего этих множеств 9 штук, то есть нечетное количество, но при этом граф не связный. \textbf{Следовательно, из нечетности количества независимых множеств не следует то, что граф связный.}
    \\
    \\ \textbf{Д12.4} При каких n в булевом кубе $Q_n$ существует остовное дерево, в котором все вершины кроме двух имеют степень 2?
    \\
    \\ \textbf{Решение: } Заметим, что дерево, в котором степени двух вершин равны 1, а степени всех остальных вершин равны 2 является в точности просто последовательностью вершин, где соседние вершины соединены ребрами. Рассмотрим индукцию по $n$ в булевом кубе $Q_n$.
    \\
    \\ \textbf{База индукции}: $n = 1$. При таком $n$ в кубе всего две вершины:
    \\
    \[
        \begin{tikzpicture}[node distance={15mm}, thick, main/.style = {draw, circle}] 
            \node[main] (1) {$1$}; 
            \node[main] (0) [right of=1] {$0$};  
            \draw[-] (1) -- (0); 
          \end{tikzpicture}
    \]
    \\ Из такого булева куба ничего не нужно удалять, он сам по себе является остовным деревом с двумя вершинами степени 1 и остальными вершинами степени 2 (таких в нем нет).
    \\
    \\ \textbf{Шаг индукции: } пусть для $n$ в графе $Q_n$ существует остовное дерево, описанное в задаче, докажем, что оно существует и для булевого куба $Q_{n + 1}$.
    \\
    \\ По определению булева куба для $n$, кажда вершина в нем является двоичным словом длины $n$. Между вершинами есть ребро, если двоичные слова в этих вершинах отличаются ровно в одной позиции.
    \\
    \\ По условию шага индукции из $Q_n$ можно выбросить часть ребер так, чтобы его вершины образовали дерево, требуемое в задаче. Это дерево задается такой последовательностью вершин \[a_1 \to a_2 \to ... \to a_{2^n}\]
    \\ При увеличении $n$ на 1 к каждой последовательности справа дописывается 0 или 1, то есть каждая вершина $Q_n$ как бы раздваивается на две разные вершины, причем эти вершины связаны ребром. Преобразуем последовательность, которую мы строили для $n$ так:
    \[
        a_10 \to a_11 \to a_21 \to a_20 \to a_30 \to ... \to a_{2^n}1 \to a_{2^n}0
    \]
    \\ То есть чтобы попасть из вершины $a_n0$ в $a_{n + 1}0$ пройдем через вершины с 1 на концах и префиксами $a_n$ и $a_{n + 1}$. В получившейся последовательности все элементы различны, так как они различаются префиксами, так как $a_k \neq a_m \ \forall (m \neq k) \in \{1, 2^n\}$, а если префиксы равны, то на концах стоят различные цифры. Так же в данной последовательности ровно $2^n \cdot 2 = 2^{n + 1}$ вершина, так как мы проходим через все вершины с 0 и 1 на конце. Следовательно в данной последовательности $2^{n + 1}$ различная вершина. Тогда эта последовательность и является остовным деревом (так как все вершины использованы) в булевом кубе $Q_{n + 1}$.
    \\
    \\ \textbf{По индукции мы доказали, что при любых $n$ в булевом кубе $Q_n$ существует остовное дерево, в котором все вершины кроме двух имеют степень 2.}
\end{document}