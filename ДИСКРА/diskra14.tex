 % Всякое, чтобы работало - все библиотеки
 \documentclass[a4paper, 12pt]{article}
 
 \usepackage[T2A]{fontenc}
 \usepackage[russian, english]{babel}
 \usepackage[utf8]{inputenc}
 \usepackage{subfiles}
 \usepackage{ucs}
 \usepackage{textcomp}
 \usepackage{array}
 \usepackage{indentfirst}
 \usepackage{amsmath}
 \usepackage{amssymb}
 \usepackage{enumerate}
 \usepackage[margin=1.5cm]{geometry}
 \usepackage{authblk}
 \usepackage{tikz}
 \usepackage{icomma}
 \usepackage{gensymb}
 \usepackage{nicematrix, tikz}
 \usepackage{tikz}

 % Всякие мат штуки дополнительные
  
 \newcommand{\F}{\mathbb{F}}
 \newcommand{\di}{\frac}
 \renewcommand{\C}{\mathbb{C}}
 \newcommand{\N} {\mathbb{N}}
 \newcommand{\Z} {\mathbb{Z}}
 \newcommand{\R} {\mathbb{R}}
 \newcommand{\Q}{\mathbb{Q}}
 \newcommand{\ord} {\mathop{\rm ord}}
 \newcommand{\Ima}{\mathop{\rm Im}}
 \newcommand{\Rea}{\mathop{\rm Re}}
 \newcommand{\rk}{\mathop{\rm rk}}
 \newcommand{\arccosh}{\mathop{\rm arccosh}}
 \newcommand{\lker}{\mathop{\rm lker}}
 \newcommand{\rker}{\mathop{\rm rker}}
 \newcommand{\tr}{\mathop{\rm tr}}
 \newcommand{\St}{\mathop{\rm St}}
 \newcommand{\Mat}{\mathop{\rm Mat}}
 \newcommand{\grad}{\mathop{\rm grad}}
 \DeclareMathOperator{\spec}{spec}
 \renewcommand{\baselinestretch}{1.5}
 \everymath{\displaystyle}
  
 % Всякое для ускорения
 \renewcommand{\r}{\right}
 \renewcommand{\l}{\left}
 \newcommand{\Lra}{\Leftrightarrow}
 \newcommand{\ra}{\rightarrow}
 \newcommand{\la}{\leftarrow}
 \newcommand{\sm}{\setminus}
 \newcommand{\lm}{\lambda}
 \newcommand{\Sum}[2]{\overset{#2}{\underset{#1}{\sum}}}
 \newcommand{\Lim}[2]{\lim\limits_{#1 \rightarrow #2}}
 \newcommand{\p}[2]{\frac{\partial #1}{\partial #2}}
  
 % Заголовки
 \newcommand{\task}[1] {\noindent \textbf{Задача #1.} \hfill}
 \newcommand{\note}[1] {\noindent \textbf{Примечание #1.} \hfill}
  
 % Пространтсва для задач
 \newenvironment{proof}[1][Доказательство]{%
 \begin{trivlist}
     \item[\hskip \labelsep {\bfseries #1:}]
     \item \hspace{15pt}
     }{
     $ \hfill\blacksquare $
 \end{trivlist}
 \hfill\break
 }
 \newenvironment{solution}[1][Решение]{%
 \begin{trivlist}
     \item[\hskip \labelsep {\bfseries #1:}]
     \item \hspace{15pt}
     }{
 \end{trivlist}
 }
  
 \newenvironment{answer}[1][Ответ]{%
 \begin{trivlist}
     \item[\hskip \labelsep {\bfseries #1:}] \hskip \labelsep
     }{
 \end{trivlist}
 \hfill
 }
 \title{Дз по линейной алгебре 2 Смирнов Тимофей 236 ПМИ}
 \author{Тимофей Смирнов}
 \date{September 2023}
 
 \begin{document}
    {\center \bf \large ДЗ по дискретной математике 14 Смирнов Тимофей 236 ПМИ}
    \\
    \\ \textbf{Д14.1} В левой доле двудольного графа 300 вершин, в правой — 400 вершин. Степени всех вершин
    в левой доле равны 4, а всех вершин в правой доле равны 3. Докажите, что в таком графе есть
    паросочетание размера 300.
    \\
    \\ \textbf{Решение: }
    \\ Назовем данный двудольный граф $G = (L, R, E)$. 
    \\ Пусть L - левая доля графа (300 вершин), а R - правая доля графа (400 вершин). Так как из всех вершин в левой доле выходит по 4 ребра, то из любого множества $ S \subset L$ выходит $4 \cdot |S|$ ребер. Все эти ребра соединены с вершинами из $G(S)$.
    \\
    \\ Из множества вершин $G(S)$ в $S$ должны идти те же ребра, что и из $S$ в $G(S)$ (то есть их не меньше, чем $4 \cdot |S|$). Но в $G(S)$ ребер всего $3 \cdot |G(S)|$. Получаем неравенство: $4 \cdot |S| \leq 3 \cdot |G(S)| \ \Rightarrow \ |S| \leq \frac{3}{4} \cdot |G(S)| \ \Rightarrow \ |S| \leq |G(S)|$. То есть мы доказали условие теоремы Холла, следовательно в данном графе $G$ существует паросочетание размера $|L| = 300$. ЧТД.
    \\
    \\ \textbf{Д14.2} В неориентированном графе на 2024 вершинах (необязательно двудольном) между любыми
    тремя вершинами есть хотя бы два ребра. Докажите, что в графе есть совершенное паросочетание (из
    1012 рёбер).
    \\
    \\ \textbf{Решение: }
    \\ Построим это паросочетание по такому правилу:
    \par I. Изначально возьмем 3 любые вершины из 2024х возможных, какие-то 2 из них будут связаны ребром по суловию. Эти вершины будут первой парой в паросочетании. Далее выбираем вершины из оставшихся 2022х.
    \par II. Теперь двействуем по двум разным сценариям, в зависимости от количества оставшихся вершин:
    \par \ \ \ \ 1. Если у нас остается 3 или более вершин для выбора пары, то выбираем из них 3 любые вершины. Среди этих 3х вершин выбираем 2 связанные ребром и добавляем в наше паросочетание. Исключаем эти 2 вершины из множества вершин для выбора новых и повторяем пункт II.
    \par \ \ \ \ \ \ \ \  2. Если у нас осталось 2 вершины, то снова получаем 2 варианта: 
    \par \ \ \ \ \ \ \ \ \ \ \ \ \ \ \  2.1 Если они связаны ребром, то добавляем их в паросочетание и завершаем работу, так как мы добавили в паросочетание все вершины. 
    \par \ \ \ \ \ \ \ \ \ \ \ \ \ \ \ \ \ \ \ \ \ 2.2 Если они не свзяаны ребром, то выбираем две связанные ребром вершины из уже добавленных в паросочетание вершин. Наши 2 вершины (которые мы еще не добавили в паросочетание) не связаны ребром, следовательно по условию задачи каждая из них будет связана с обеими выбранными вершинами. Тогда мы можем соединить первую из вершин, с которой мы работаем с первой выбранной, а вторую со второй, затем удилить ребро между найденной в паросочетании парой вершин, а вместой этой пары добавить 2 новые, только что полученные. В итоге мы получим паросочетание, использовав все вершины графа.
    \\
    \\ Иллюстрация к пункту 2.2:
    \\ 1 шаг: Ннаходим пару уже добавленных в паросочетание вершин: 1 и 2 (желтыми линиями обозначены ребра, которые мы не использовали в паросочетании)
    \[
        \begin{tikzpicture}[node distance={15mm}, thick, main/.style = {draw, circle}] 
            \node[main] (1) {$1$}; 
            \node[main] (2) [above of=1] {$2$};  
            \node[main] (3) [right of=2] {$3$}; 
            \node[main] (4) [below of=3] {$4$}; 
            \draw[-] (1) -- (2); 
            \draw[yellow] (2) -- (3);
            \draw[yellow] (2) -- (4);
            \draw[yellow] (1) -- (3);
            \draw[yellow] (1) -- (4);
        \end{tikzpicture}
    \]
    \\ 2 шаг: Разрываем связь между вершиной 3 и 4, добавляем связи 3-5 и 4-6
    \[
        \begin{tikzpicture}[node distance={15mm}, thick, main/.style = {draw, circle}] 
            \node[main] (1) {$1$}; 
            \node[main] (2) [above of=1] {$2$}; 
            \node[main] (3) [right of=2] {$3$}; 
            \node[main] (4) [below of=3] {$4$}; ; 
            \draw[-] (1) -- (3); 
            \draw[-] (2) -- (4);
            \draw[yellow] (2) -- (3);
            \draw[yellow] (1) -- (4);
            \draw[yellow] (1) -- (2); 
        \end{tikzpicture} 
    \]
    \\
    \\ После выполнения алгоритма мы получим паросочетание из всех вершин графа, следовательно оно будет совершенным паросочетанием.
    \\
    \\ \textbf{Д14.3} В неориентированном графе на 101 вершине есть независимое множества размера 52. Докажите,
    что в этом графе нет паросочетания размера 50.
    \\
    \\ \textbf{Решение: }
    \\ Пусть $G$ - наш неориентированный граф нв 101 вершине.
    \\ На лекции мы узнали, что $\tau(G) \geq \mu(G)$. Так же из лекции мы знаем, что $\tau(G) = n - \alpha(G) = 101 - \alpha(G)$.
    \\ Так как в нашем графе есть независимое множество размера 52, то $\alpha(G) \geq 52 \ \Rightarrow \  101 - 52 \geq 101 - \alpha(G) = \tau(G) \ \Rightarrow \ \tau(G) \leq 49$. 
    \\ А так как $\tau(G) \geq \mu(G) \ \Rightarrow \ \mu(G) \leq 49$. То есть в графе $G$ нет паросочетания размера 50, так как размер максимального паросочетания меньше 50.
    \\
    \\ \textbf{Д14.4} В неориентированном графе на n вершинах есть вершинное покрытие размера 10. Докажите,
    что в таком графе нет простого пути длины 21. (В простом пути все вершины разные, длина пути —
    количество рёбер в нём.)
    \\
    \\ \textbf{Решение: }
    \\ Так как в простом пути все вершины различны, то в пути из 21 ребра, нам необходимо 22 разичные вершины (в самом графе их может быть больше, но точне не меньше).
    \\ 
    \\ Чтобы выбрать вершинное покрытие необходимо задействовать все ребра по определению. 
    \\
    \\ Чтобы задействовать все ребра из нашего простого пути, можно воспользоваться только вершинами из этого пути, так как другие вершины графа не охватят все ребра. 
    \\
    \\ Чтобы охватить каждое из 21го ребра в простом пути, необходимо брать вершины через одну. Если мы возьмем больше вершин, то нам будет хуже, а если меньше, то мы не охватим все ребра, так как найдется хотя бы одно ребро, у которого вершины на обеих концах не взяты. Итак, из 22х вершин, выстроенных в линию, через одну мы можем взять 11. Следоваательно в нашем вершинном покрытии будет минимум 11 вершин. Получаем противоречие с условием, следовательно в данном графе не может быть простого пути длины 21.
\end{document}