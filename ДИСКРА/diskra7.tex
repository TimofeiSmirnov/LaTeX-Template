 % Всякое, чтобы работало - все библиотеки
 \documentclass[a4paper, 12pt]{article}
 
 \usepackage[T2A]{fontenc}
 \usepackage[russian, english]{babel}
 \usepackage[utf8]{inputenc}
 \usepackage{subfiles}
 \usepackage{ucs}
 \usepackage{textcomp}
 \usepackage{array}
 \usepackage{indentfirst}
 \usepackage{amsmath}
 \usepackage{amssymb}
 \usepackage{enumerate}
 \usepackage[margin=1.5cm]{geometry}
 \usepackage{authblk}
 \usepackage{tikz}
 \usepackage{icomma}
 \usepackage{gensymb}
 \usepackage{nicematrix, tikz}
 \setcounter{MaxMatrixCols}{15}
  
 % Всякие мат штуки дополнительные
  
 \newcommand{\F}{\mathbb{F}}
 \newcommand{\di}{\frac}
 \renewcommand{\C}{\mathbb{C}}
 \newcommand{\N} {\mathbb{N}}
 \newcommand{\Z} {\mathbb{Z}}
 \newcommand{\R} {\mathbb{R}}
 \newcommand{\Q}{\mathbb{Q}}
 \newcommand{\ord} {\mathop{\rm ord}}
 \newcommand{\Ima}{\mathop{\rm Im}}
 \newcommand{\Rea}{\mathop{\rm Re}}
 \newcommand{\rk}{\mathop{\rm rk}}
 \newcommand{\arccosh}{\mathop{\rm arccosh}}
 \newcommand{\lker}{\mathop{\rm lker}}
 \newcommand{\rker}{\mathop{\rm rker}}
 \newcommand{\tr}{\mathop{\rm tr}}
 \newcommand{\St}{\mathop{\rm St}}
 \newcommand{\Mat}{\mathop{\rm Mat}}
 \newcommand{\grad}{\mathop{\rm grad}}
 \DeclareMathOperator{\spec}{spec}
 \renewcommand{\baselinestretch}{1.5}
 \everymath{\displaystyle}
  
 % Всякое для ускорения
 \renewcommand{\r}{\right}
 \renewcommand{\l}{\left}
 \newcommand{\Lra}{\Leftrightarrow}
 \newcommand{\ra}{\rightarrow}
 \newcommand{\la}{\leftarrow}
 \newcommand{\sm}{\setminus}
 \newcommand{\lm}{\lambda}
 \newcommand{\Sum}[2]{\overset{#2}{\underset{#1}{\sum}}}
 \newcommand{\Lim}[2]{\lim\limits_{#1 \rightarrow #2}}
 \newcommand{\p}[2]{\frac{\partial #1}{\partial #2}}
  
 % Заголовки
 \newcommand{\task}[1] {\noindent \textbf{Задача #1.} \hfill}
 \newcommand{\note}[1] {\noindent \textbf{Примечание #1.} \hfill}
  
 % Пространтсва для задач
 \newenvironment{proof}[1][Доказательство]{%
 \begin{trivlist}
     \item[\hskip \labelsep {\bfseries #1:}]
     \item \hspace{15pt}
     }{
     $ \hfill\blacksquare $
 \end{trivlist}
 \hfill\break
 }
 \newenvironment{solution}[1][Решение]{%
 \begin{trivlist}
     \item[\hskip \labelsep {\bfseries #1:}]
     \item \hspace{15pt}
     }{
 \end{trivlist}
 }
  
 \newenvironment{answer}[1][Ответ]{%
 \begin{trivlist}
     \item[\hskip \labelsep {\bfseries #1:}] \hskip \labelsep
     }{
 \end{trivlist}
 \hfill
 }
 \title{Дз по линейной алгебре 2 Смирнов Тимофей 236 ПМИ}
 \author{Тимофей Смирнов}
 \date{September 2023}
\begin{document}
    {\center \bf \large ДЗ по дискретной математике 7 Смирнов Тимофей 236 ПМИ}
    \\
    \\ \textbf{Д7.1} В группе 40 туристов. Из них 20 человек знают английский язык, 15 — французский, 11 — испанский. Английский и французский знают 7 человек, английский и испанский — 5, французский и
    испанский — 3. Двое туристов знают все три языка. Сколько человек в группе не знает ни одного из
    этих языков?
    \\
    \\
    \\ \textbf{Рещение: } Посчитаем по формуле включения-исключения. Пусть множество $A  - $ люди, знающие английский, $B - $ люди, знающие испанский и $C -$ люди, знающие испанский.
    \\
    \\ $|A \cup B \cup C|$ - люди, знающие хоть какой-то их трех языков.
    \\ $|A| = 20$ Столько людей знают английский.
    \\ $|B| = 15$ Столько людей знают французский.
    \\ $|C| = 11$ Столько людей знают испанский.
    \\ $|A \cap B| = 7$ - Столько человек знают английский и французский.
    \\ $|A \cap C| = 5$ - Столько человек знают английский и испанский.
    \\ $|B \cap C| = 3$ - Столько человек знают францухский и испанский.
    \\ $|A \cap B \cap C| = 2$ - Столько людей знает все 3 языка.
    \\
    \\ $|A \cup B \cup C| = |A| + |B| + |C| - |A \cap B| - |A \cap C| - |B \cap C| + |A \cap B \cap C| = 20 + 15 + 11 - 7 - 5 - 3 + 2 = 33$.
    \\
    \\ Получаем, что из сорока человек в группе только 33 человека знают хоть какой-то язык. Получается, что ни одного языка не знают $40 - 33 = 7$ человек.
    \\
    \par \ \ \ \ \textbf{Ответ: } 7.
    \\
    \\
    \\ \textbf{Д7.2} Есть 3 гвоздики, 4 розы и 5 тюльпанов. Сколькими способами можно составить букет из 7 цветов,
    используя имеющиеся цветы? (Цветы одного сорта считаем одинаковыми.) Ответом должно быть число
    в десятичной записи.
    \\
    \\
    \\ \textbf{Рещение: } Посчитаем по формуле включения-исключения. Пусть множество $A  - $ люди, знающие английский, $B - $ люди, знающие испанский и $C -$ люди, знающие испанский.
    \\
    \\ Пеберем все возможные варианты. Так как всего в сумме цветов 12, а тюльпанов из них 5, то можно перебирать по тюльпанам, начиная с 0:
    \\
    \\ Для сокращения вмместо Тюльпанов я буду использовать букву Т, вместе гвоздик - Г, вместо роз - Р.
    \\
    \par \ \ \ \ 1). Пусть мы берем 0 Т, тогда мы должны взять (3 Г и 4 Р). Всего 1 вариант.
    \par \ \ \ \ 2). Пусть мы берем 1 Т, тогда мы можем взять либо (2 Г и 4 Р), либо (3 Г и 3 Р). Всего 2 варианта.
    \par \ \ \ \ 3). Пусть мы берем 2 Т, тогда мы можем взять (1 Г и 4 Р), (2 Г и 3 Р) или (3 Г и 2 Р). Всего 3 варианта.
    \par \ \ \ \ 4). Пусть мы берем 3 Т, тогда мы можем взять (0 Г и 4 Р), (1 Г и 3 Р), (2 Г и 2 Р) или (3 Г и 1 Р). Всего 4 варианта.
    \par \ \ \ \ 5). Пусть мы берем 4 Т, тогда мы можем взять (0 Г и 3 Р), (1 Г и 2 Р), (2 Г и 1 Р) или (3 Г и 0 Р). Всего 4 варианта.
    \par \ \ \ \ 6). Пусть мы берем 5 Т, тогда мы можем взять (0 Г и 2 Р), (1 Г и 1 Р) или (2 Г и 0 Р). Всего 3 варианта.
    \\
    \\ Получаем ответ: $1 + 2 + 3 + 4 + 4 + 3 = 17$
    \\
    \\ \textbf{Ответ: } 17 вариантов собрать букет.
    \\
    \\
    \\ \textbf{Д7.3} Сколько двоичных слов длины 12 содержат подслово 1100? Подслово — это последовательность
    стоящих подряд символов. Ответ должен быть целым числом в десятичной записи.
    \\
    \\ 1). Всего вариантов получить слово с подсловом 1100 будет $9 \cdot 2^8  = 2304$, так как позицию для первой 
    единицы в подслове 1100 мы можем выбрать девятью способами (последние три позиции из 12 не подходят, так как в слове 1100 четыре цифры).
    \\
    \\Но некоторые слова мы сочитали по 2 раза. Например, слова типа 1100****1100 могли посчитаться по 2 раза. Чтобы этого избежать нам нужно вычесть количество слов c двумя подсловами 1100. Таких слов у нас $C_6^2 \cdot 2^4 = 240$ ($C_6^2$ получилось, так как я заменил первую 1100 на А и  вторую 1100 на В (А и В одинаковые) и выбрал из возможных 12 - 4 + 2 = 6 позиций возможные позиции для них, а на $2^4$ мы умножаем, чтобы рассмотреть все варианты поставить 0 или 1 на оставшихся 4х позициях). 
    \\
    \\Но при вычитании слов с двумя подсловами 1100 мы так же вычли все варианты получить слово типа 110011001100, поэтому в конце мы должны прибавить это слово.
    \\
    \\ Получаем формулу $2304 - 240 + 1 = 2065$
    \\
    \par \ \ \ \ \textbf{Ответ: } 2065.
    \end{document}